%% USPSC-TCC-pre-textual-OUTROS.tex
%% Camandos para definição do tipo de documento (tese ou dissertação), área de concentração, opção, preâmbulo, titulação 
%% referentes aos Programas de Pós-Graduação
\instituicao{Faculdade de Filosofia, Ciências e Letras de Ribeirão Preto, Universidade de São Paulo}
\unidade{\MakeUppercase{Faculdade de Filosofia, Ciências e Letras de\\
Ribeirão Preto}}
\unidademin{Faculdade de Filosofia, Ciências e Letras de Ribeirão Preto}
\universidademin{Universidade de S\~ao Paulo}

% A EESC não inclui a nota "Versão original", portanto o comando abaixo não tem a mensagem entre {}
\notafolharosto{ }
%Para a versão corrigida tire a % do comando/declaração abaixo e inclua uma % antes do comando acima
%\notafolharosto{VERS\~AO CORRIGIDA}
% ---
% dados complementares para CAPA e FOLHA DE ROSTO
% ---
\universidade{UNIVERSIDADE DE S\~AO PAULO}
\titulo{Estudo Numérico de Equações em Meios Porosos}
\tituloresumo{Estudo Numérico de Equações em Meios Porosos}
\titleabstract{Numerical Study of Porous Media Equations}
\autor{Henrique Casellato Vitorio Rodrigues da Costa}
\autorficha{Costa, Henrique C. V. R.}
\autorabr{COSTA, H. C. V. R.}

\cutter{S856m}
% Para gerar a ficha catalográfica sem o Código Cutter, basta 
% incluir uma % na linha acima e tirar a % da linha abaixo
%\cutter{ }

\local{Ribeirão Preto}
\data{2025}
% Quando for Orientador, basta incluir uma % antes do comando abaixo
\renewcommand{\orientadorname}{Orientador:}
% Quando for Coorientadora, basta tirar a % do comando abaixo
%\newcommand{\coorientadorname}{Coorientador:}
\orientador{Prof. Dr. Nikolai Vasilievich Chemetov.}
%\orientadorcorpoficha{orientadora Elisa Gon\c{c}alves Rodrigues}
%\orientadorficha{Rodrigues, Elisa Gon\c{c}alves, orient}
%Se houver co-orientador, inclua % antes das duas linhas (antes dos comandos \orientadorcorpoficha e \orientadorficha) 
%          e tire a % antes dos 3 comandos abaixo
\orientadorcorpoficha{Orientador Nikolai Vasilievich Chemetov}
\orientadorficha{Chemetov, Nikolai Vasilievich}

\notaautorizacao{AUTORIZO A REPRODU\c{C}\~AO E DIVULGA\c{C}\~AO TOTAL OU PARCIAL DESTE TRABALHO, POR QUALQUER MEIO CONVENCIONAL OU ELETR\^ONICO PARA FINS DE ESTUDO E PESQUISA, DESDE QUE CITADA A FONTE.}
\notabib{~  ~}

\newcommand{\programa}[1]{     	
% Outros
	\tipotrabalho{Monografia (Trabalho de Conclus\~ao de Curso)}
	\tipotrabalhoabs{Monograph (Conclusion Course Paper)}
	%\area{Nome da área}
	%\opcao{Nome da Opção}
	% O preambulo deve conter o tipo do trabalho, o objetivo, 
	% o nome da instituição, a área de concentração e opção quando houver
	\preambulo{Monografia apresentada ao Curso de Ciência da Computação, da Unidade FFCLRP da Universidade de S\~ao Paulo, como parte dos requisitos para obten\c{c}\~ao do t\'itulo de Cientista da Computação.}
	\notaficha{Monografia (Gradua\c{c}\~ao em Ciência da Computação)}	
	}
