%% USPSC-Agradecimentos.tex
\begin{agradecimentos}
A universidade teria sido um moroso tédio, temperado com dificuldades mil, caso não tivesse encontrado aqui minha família, que, pelas conversas de corredor, pelos passeios até o bandejão, ou pelas intensas discussões, puderam me mostrar um caminho mais ameno. Gostaria de, primeiramente, agradecer aos meus queridíssimos amigos Ana Júlia Ticianeli, Eduardo Campos e João Vitor Lopes, que me acompanharam desde as primeiras semanas até este final, que ofereceram alento nos momentos mais árduos e celebraram minhas festas aos mais extensos risos. Nosso quarteto tornou-se minha casa  nesta cidade tão nova. Além, gostaria de agradecer a tantos primos que pude colecionar, aos meus amigos Letícia Landim, Edmara Viana, Maria José Ayala; aos meus camaradas do Centro Estudantil da Computação, da Representação Discente, à minha família Amendoim; e por fim, às amigas e professoras do circo e do ballet.

Meus recessos, pude me aconchegar em Brasília acompanhado de um grupo tão meigo, que a distância somente enraizou mais nossos laços. Meus amigos do peito sanguíneo, os mais amados e guardados \textit{ad infinitum} no átrio, aos meus irmãos Eduarda Souza, Felipe Cabral, João Luiz Matos, Andrey Fujikawa, Enzo Candeias, Meireles Amancio, Charles Mariano, Nathalia Oliveira e tantos, tantos e mais tantos outros que pude acolher tão docemente. Aos meus amigos mais distantes, minha irmã Maria Beatriz de Crasto em Pernambuco e minha querida amiga Mariana Rocha em Portugal, das quais me orgulho tão demasiadamente. Por fim, sou profundamente grato à pletora de professores do meu Ensino Médio, mas especialmente ao matemático Eliomar Caetano, à artista e tatuadora Isabela Alves e à historiadora Danielle Magalhões, que me orientaram nos tão incômodos meandros da puerilidade. 

A graça de poder somente me dedicar aos estudos, pude somente dado aos esforços integrais de meus queridos pais, o grande engenheiro João Batista Rodrigues e a incrível artista plástica Cybele Casellato, assim como do apoio de minha irmã caçula Luiza Casellato. Minha família nuclear foi minha base fundamental, pois me assegurou o que muito é estruturalmente furtado da realidade de grande parte dos brasileiros: desde o tempo para estudar até os recursos para me deslocar de Brasília a Ribeirão, ou morar com grande qualidade de vida. Todos os meus períodos de paz ou de turbulência foram carinhosamente agasalhados por minha formidável psicóloga Mônica Bueno, cuja inteligência foi imprescindível para meu esqueleto psíquico. Ainda por cima, gostaria de agradecer aos meus tios-avós Marta Cazelato e Manoel Pires, além de minhas queridíssimas primas Manuela, Giovanna e Amanda, as memórias que tenho de vocês serão sempre as mais divertidas.

Por último, gostaria de demonstrar minha profunda gratidão por meus antigos orientadores Tiago de Carvalho e Hermano Neto, e por meu atual orientador Nikolai Chemetov, por toda a confiança, todos os conselhos e todas as centenas de oportunidades que pude aproveitar. Ademais, gostaria de agradecer a todos os professores e aos funcionários do Departamento de Computação e Matemática da Faculdade, especialmente aos professores Vanessa Rolnik, Katia de Azevedo, Joaquim Felipe, Renato Tinós e Clever Ricardo, que foram responsáveis por endireitar meu caminho dentro da universidade.

Meu trajeto foi recheado pela sorte que tive ao encontrar tantas pessoas, tantos colegas, tantos amigos, que o acaso gentilmente me apresentou. Serei, por isso, eternamente grato.

\end{agradecimentos}
% ---