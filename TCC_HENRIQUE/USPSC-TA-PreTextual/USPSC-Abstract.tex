%% USPSC-Abstract.tex
%\autor{Silva, M. J.}
\begin{resumo}[Abstract]
 \begin{otherlanguage*}{english}
	\begin{flushleft} 
		\setlength{\absparsep}{0pt} % ajusta o espaçamento dos parágrafos do resumo		
 		\SingleSpacing
        \justify{\imprimirautorabr~~\textbf{\imprimirtitleabstract}.	\imprimirdata.  \pageref{LastPage} p. 
		\imprimirtipotrabalhoabs~-~\imprimirinstituicao, \imprimirlocal, 	\imprimirdata. }
 	\end{flushleft}
	\OnehalfSpacing 
   This monograph presents a numerical study of elliptic and hyperbolic partial differential equations in the context of single-phase flow in porous media and contaminant transport. Finite difference methods were implemented and analyzed, but the main focus was on finite volume methods, in order to solve one- and two-dimensional boundary value and flow problems, considering both homogeneous and heterogeneous media. The approach included the conservative discretization of the pressure equation and the transport equation, exploring various matrix factorization techniques and \textit{upwind} schemes for flux approximation. The numerical results demonstrate the effectiveness of the methods in simulating pressure fields and contaminant propagation, with truncation errors on the order of $O(h^2)$ and compliance with the CFL condition. It is concluded that the applied numerical techniques are robust and suitable for computational modeling in porous media, providing a foundation for more complex studies, such as multiphase flow and hydro-mechanical coupling.

   \vspace{\onelineskip}
 
   \noindent 
   \textbf{Keywords}: Numerical Methods. Porous Media. Darcy's Law. Finite Volumes. Partial Differential Equations. Single-Phase Flow. Contaminant Transport. Computational Simulation. 
 \end{otherlanguage*}
\end{resumo}
