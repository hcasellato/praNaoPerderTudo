%% USPSC-Resumo.tex
\setlength{\absparsep}{18pt} % ajusta o espaçamento dos parágrafos do resumo		
\begin{resumo}
	\begin{flushleft} 
			\setlength{\absparsep}{0pt} % ajusta o espaçamento da referência	
			\SingleSpacing 
			\justify{\imprimirautorabr~~\textbf{\imprimirtituloresumo}.	\imprimirdata. \pageref{LastPage} p.
			\imprimirtipotrabalho~-~\imprimirinstituicao, \imprimirlocal, \imprimirdata. }
 	\end{flushleft}
\OnehalfSpacing 			
	Esta monografia apresenta um estudo numérico de equações diferenciais parciais elípticas e hiperbólicas, no contexto de escoamentos monofásicos em meios porosos e transporte de contaminantes. Foram implementados e analisados métodos de diferenças finitas, mas principalmente de volumes finitos, a fim de resolver os problemas de valores de contorno e de escoamento, unidimensionais e bidimensionais; com meios homogêneos e heterogêneos. A abordagem incluiu a discretização conservativa da equação de pressão e da equação de transporte, explorando diversas técnicas de fatorações matriciais e esquemas \textit{upwind} para aproximação de fluxos. Os resultados numéricos demonstram a eficácia dos métodos na simulação de campos de pressão e na propagação de contaminantes, com erros de truncamento da ordem de $O(h^2)$ e conformidade com a condição CFL. Conclui-se que as técnicas numéricas aplicadas são robustas e adequadas para modelagem computacional em meios porosos, oferecendo bases para estudos mais complexos, como escoamentos multifásicos e acoplamento hidromecânico.
	
	\vspace{\onelineskip}

	\noindent 
 \textbf{Palavras-chave}: Métodos Numéricos. Meios Porosos. Lei de Darcy. Volumes Finitos. Equações Diferenciais Parciais. Escoamento Monofásico. Transporte de Contaminantes. Simulação Computacional.
\end{resumo}