%% USPSC-Cap3-Conclusao.tex
% ---
% Conclusão
% ---
\chapter{Conclusão}
% ---

O objetivo principal deste trabalho de conclusão de curso consistiu em estudar \textit{métodos numéricos para equações diferenciais} com aplicações em \textit{meios porosos}, a fim de formar o aluno com uma base sólida e estruturada em mecânica do fluidos computacional e prepará-lo para o prosseguimento de seus estudos na pós-graduação. Considera-se que este objetivo foi alcançado, pois, conforme elaborado nesta monografia, foi possível abordar desde a teoria de equações diferenciais parciais e problemas de valores de contorno, até a modelagem de escoamentos monofásicos por meio do método de volumes finitos.

Além de revisitar e aprofundar tópicos já presentes na formação em Ciência da Computação -- como álgebra linear, métodos numéricos e análise de algoritmos -- este trabalho permitiu o estudo de conteúdos não contemplados na graduação, dentre os quais destacam-se o método de volumes finitos, a teoria de EDPs elípticas e hiperbólicas, e a simulação de escoamentos monofásicos.

Os resultados presentes na monografia demonstram a robustez do método de volumes finitos, tanto na discretização de equações elípticas quanto para as hiperbólicas; além de discutir, durante sua implementação no caso unidimensinal, três técnicas de fatoração matricial para a resolução dos sistemas lineares resultantes. Os erros de truncamentos obtidos foram consistentes com a literatura estudada, os exemplos numéricos incluíram casos de uma e duas dimensões, além de contemplarem diferentes níveis de complexidade e perfis de permeabilidade. As simulações permitiram a visualização dos campos de pressão e de velocidades, assim como da dispersão de contaminantes, então validando a metodologia adotada e reforçando sua aplicabilidade em problemas reais de engenharia de reservatórios e hidrologia.

Cabe ressaltar que o trabalho limitou-se ao tratamento de escoamentos monofásicos e incompressíveis, com equações governantes simplificadas e em geometrias uniformes de uma e duas dimensões. Não foram abordados fenômenos de escoamentos multifásicos (como \textit{óleo-água}, relevantes para a indústria do petróleo), de malhas não estruturadas, de acoplamentos hidromecânicos, ou ainda (no contexto das tecnologias da computação) da paralelização de CPU ou GPU. Apesar disso, o estudo realizado serve como base promissora para investigações futuras, a exemplo de estudos envolvendo meios porosos não saturados e extensões para problemas tridimensionais e com termos de transporte reativo.

Por fim, este trabalho não apenas cumpriu seu propósito formativo, como também abriu caminho para a continuação da pesquisa acadêmica em matemática aplicada e computacional, áreas de grande relevância na academia, indústria e projetos de preservação ambiental.