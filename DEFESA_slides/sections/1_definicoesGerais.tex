\section{Definições gerais} % Seções são adicionadas para organizar sua apresentação em blocos discretos, todas as seções e subseções são automaticamente exibidas no índice como uma visão geral da apresentação, mas NÃO são exibidas como slides separados.

%----------------------------------------------------------------------------------------

\begin{frame}
  \frametitle{O que é um meio poroso?}
  Um meio poroso é caracterizado por sua \textbf{porosidade} $\phi$, a fração entre o volume dos espaços vazios $V_p$ e o total $V_t$:
  \[
    \phi = \frac{V_p}{V_t},
  \]
  portanto sendo uma grandeza adimensional entre zero e um. 
  \begin{figure}[H]
    \centering
    % \begin{subfigure}[b]{0.32\textwidth}
    %   \centering
    %   \includegraphics[width=\textwidth]{img/Porousceramic.jpeg}
    %   \caption{Cerâmica \\
    %   {\tiny Fonte: Biofilter tech / Wikimedia Commons (CC BY-SA 3.0)}}
    % \end{subfigure}
    \begin{subfigure}[b]{0.32\textwidth}
      \centering
      \includegraphics[width=\textwidth]{img/ElTorcal0408.jpg}
      \caption{Calcário {\tiny (Torcal de Antequera)} \\
      {\tiny Fonte: Fernando Domínguez Cerejido / Wikimedia Commons (CC BY-SA 4.0)}}
    \end{subfigure}
    \hfill
    \begin{subfigure}[b]{0.32\textwidth}
      \centering
      \includegraphics[width=\textwidth]{img/Groundwater.png}
      \caption{Aquífero subterrâneo \\
      {\tiny Fonte: T.C. Winter et al., U.S. Geological Survey (Domínio Público)}}
    \end{subfigure}
    \hfill
    \begin{subfigure}[b]{0.32\textwidth}
      \centering
      \includegraphics[width=\textwidth]{img/Oilwells.jpg}
      \caption{Reservatório de petróleo \\
      {\tiny Fonte: G.H. Eldridge (c. 1905) / NOAA (Domínio Público)}}
    \end{subfigure}
    \caption{Exemplos de meios porosos.}
  \end{figure}
\end{frame}

%----------------------------------------------------------------------------------------

% \begin{frame}
%   \frametitle{Por que estudar meios porosos?}
%   Lorem ipsum dolor sit amet, consectetur adipiscing elit:
% \end{frame}

%----------------------------------------------------------------------------------------

\begin{frame}
	\frametitle{Compressibilidade e viscosidade}
  \textbf{Meios} podem ser considerados:
  \begin{itemize}
    \justifying
    \item \textbf{Incompressíveis}, se têm porosidade \textit{estática}; e
    \item \textbf{Compressíveis}, se têm porosidade \textit{dinâmica}.
  \end{itemize}

  Os \textbf{escoamentos}, por sua vez, podem ser classificados como:
  \begin{itemize}
    \justifying
    \item \textbf{Incompressíveis}: possui compressibilidade zero, o que implica uma massa específica constante independentemente da pressão;
    \item \textbf{Levemente compressíveis}: possui uma compressibilidade baixa e constante, com sua massa específica variando linearmente com a pressão; ou
    \item \textbf{Compressíveis}: possui compressibilidade alta e sua massa específica cresce com a pressão, tendendo a se estabilizar em altas pressões.
  \end{itemize}

  A \textbf{viscosidade} é a \textit{medida da resistência de um fluido ao próprio escoamento}. Os gases apresentam baixa viscosidade, e os fluidos mais densos (óleo) apresentam alta viscosidade.
\end{frame}

%----------------------------------------------------------------------------------------

\begin{frame}
  \frametitle{Permeabilidade do meio}
  A \textbf{permeabilidade} $K$ representa a \textit{habilidade} do meio poroso em \textit{transmitir um fluido através dos seus poros interconectados}, quando completamente saturado do mesmo. Quando somente dá um único fluido no meio poroso, esta propriedade recebe o nome de \textbf{permeabilidade absoluta}.

  A permeabilidade é dada por um tensor \textit{simétrico positivo definido}:
  \[
    K = 
    \begin{bmatrix}
      K_{xx} && K_{xy} && K_{xz} \\
      K_{yx} && K_{yy} && K_{yz} \\
      K_{zx} && K_{zy} && K_{zz}
    \end{bmatrix},
  \]
  caso ela seja \textbf{isotrópica}, $K$ é um escalar.

  Quando $K$ é espacialmente constante o meio se denomina \textbf{homogêneo} e caso contrário, \textbf{heterogêneo}.
\end{frame}

%----------------------------------------------------------------------------------------

% \begin{frame}
%   \frametitle{Viscosidade}
%   A \textbf{viscosidade} é a \textit{medida da resistência de um fluido ao próprio escoamento}. Os gases, que possuem moléculas distanciadas, apresentam baixa resistência ao escoamento e portanto, baixa viscosidade. Por outro lado, fluidos mais densos apresentam alta resistência ao escoamento, e consequentemente alta viscosidade.

%   \begin{itemize}
%     \justifying
%     \item No caso de \textbf{água} (um escoamento incompressível), vale notar que neste caso a viscosidade não varia quando a pressão aumenta;
%     \item Já no caso do \textbf{óleo}, a variação da viscosidade ocorre, pois o óleo pode transferir parte de sua massa para a forma de gás (escoamento compressível).
%   \end{itemize}

%   Vale ressaltar que é possível medir a variação da viscosidade em relação à pressão do reservatório, considerando o efeito da pressão nas massas específicas dos fluidos.
% \end{frame}

%----------------------------------------------------------------------------------------

\begin{frame}
  \frametitle{Lei de Darcy}
  Para introduzir a modelagem de escoamentos de fluidos, é preciso antes entender a \textbf{velocidade de Darcy}. Esta equação advém de estudos do fluxo físico da água em um filtro vertical de areia feitos por Henry Darcy em 1856.
  \begin{figure}[H]
    \centering
    \includegraphics[width=.5\textwidth]{img/experimento_Darcy.png}
    \caption{Experimento de Darcy, SOUZA e ROCHA (2022, p. 16).}
  \end{figure}
\end{frame}

%----------------------------------------------------------------------------------------

% \begin{frame}
%   \frametitle{Lei de Darcy}
%   Sua \textbf{altura hidráulica} $h$ (com $h_t$ no topo e $h_b$ na base) em relação a um ponto fixo $z$ dada por
%   \[
%     h = -\frac{p}{\rho g} + z,
%   \]
%   onde $g$ é a magnitude da aceleração da gravidade, $p$ é a pressão de água e $\rho$ é a massa específica da água. A taxa de fluxo $Q$ pode ser equacionada por
%   \[
%     \frac{Q}{A} = \mathbf{k}\frac{h_t - h_b}{L}\check{e},
%   \]
%   onde $\check{e}$ é o vetor direção do escoamento e $\mathbf{k}$ a condutividade hidráulica, que (nesse contexto) é dada por
%   \[
%     \mathbf{k} = \frac{\rho g K}{\mu},
%   \]
%   sendo $\mu$ é a viscosidade do fluido.
% \end{frame}

%----------------------------------------------------------------------------------------

\begin{frame}
  \frametitle{Lei de Darcy}
  Este fluxo representa o volume total do fluido pela área total por tempo, sendo também conhecido como \textbf{velocidade de Darcy} dada por:
  \begin{equation}\label{leiDeDarcy}
    \begin{aligned}
      u = \frac{Q}{A} = \mathbf{k}\frac{h_t - h_b}{L}\check{e} &= \frac{\rho g K}{\mu}\nabla h \\
      &= \frac{\rho g K}{\mu}\nabla \left(-\frac{p}{\rho g} + z\right) \\
      &= -\frac{K}{\mu} (\nabla p - \rho g \nabla z),
    \end{aligned}
  \end{equation}
  sendo $z$ a profundidade. Esta equação (\theequation)\ é conhecida como \textbf{Lei de Darcy} e representa a conservação de quantidade de movimento, na qual duas forças governam o fluxo: a pressão e a gravidade.
\end{frame}

%----------------------------------------------------------------------------------------