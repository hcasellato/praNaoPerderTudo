%%%%%%%%%%%%%%%%%%%%%%%%%%%%%%%%%%%%%%%%%
% Beamer Presentation - LaTeX Template
% Version 2.0 (March 8, 2022)
% Original Template: https://www.LaTeXTemplates.com
% Author: Vel (vel@latextemplates.com)
% License: CC BY-NC-SA 4.0

% Este modelo de apresentação foi 
% criado a partir do modelo de Giovanni Spadaro.
% Disponível em: https://github.com/Giovo17/presentation-template-unict-lm-data
%
% Adaptado por Lucas Amaral Taylor para criar uma versão especial 
% para os alunos de Matemática e Estatística da USP (IME-USP).
% Disponível em: https://github.com/lucasamtaylor01/IME-template
%%%%%%%%%%%%%%%%%%%%%%%%%%%%%%%%%%%%%%%%%

%----------------------------------------------------------------------------------------
% CLASSE DO DOCUMENTO E CONFIGURAÇÕES BÁSICAS
%----------------------------------------------------------------------------------------
\documentclass[
    11pt,               % Tamanho padrão da fonte
    % t,                % Alinhar verticalmente ao topo
    %aspectratio=169,   % Defirannir proporção 16:9
]{beamer}
\graphicspath{{img/}}         % Define o diretório das imagens

\usepackage{mathptmx}
\usepackage[uprightGreek, libertine]{newtxmath}
%\usepackage{helvet}
\renewcommand{\familydefault}{\rmdefault}

\theoremstyle{plain} % Estilo padrão para teoremas, proposições, etc.
\newtheorem{teorema}{Teorema} % Numeração por seção
\newtheorem{proposicao}[teorema]{Proposição} % Mesma numeração que teoremas
\newtheorem{lema}[teorema]{Lema}
\newtheorem{corolario}[teorema]{Corolário}

\theoremstyle{definition} % Estilo para definições, exemplos
\newtheorem{definicao}{Definição}
\newtheorem{exemplo}{Exemplo}
\newtheorem{exercicio}{Exercício}

\theoremstyle{remark} % Estilo para observações
\newtheorem*{observacao}{Observação} % Asterisco remove numeração
\newtheorem*{notacao}{Notação}

%----------------------------------------------------------------------------------------
% PACOTES NECESSÁRIOS
%----------------------------------------------------------------------------------------
\usepackage{
    booktabs,     % Melhora a aparência das linhas em tabelas
    subcaption    % Suporte para subfiguras
}
\usepackage{times}
\input{config/code_langs}       % Importa configurações para highlight de código

%----------------------------------------------------------------------------------------
% CONFIGURAÇÃO DO TEMA
%----------------------------------------------------------------------------------------
\usetheme{Madrid}
\useinnertheme{circles}
\useoutertheme{miniframes} 
\setbeamertemplate{headline}[default]
\setbeamertemplate{navigation symbols}{}
\usefonttheme{serif}

% === DEEP INDIGO MONOCHROMATIC THEME ===
% Elegant, academic, and modern — perfect for lectures or defenses.

% --- COLOR DEFINITIONS ---
\definecolor{primaryColor}{RGB}{35, 40, 90}        % Deep indigo (main)
\definecolor{secondaryColor}{RGB}{60, 70, 130}     % Medium indigo (headers)
\definecolor{accentColor}{RGB}{120, 130, 190}      % Light indigo (accents)
\definecolor{backgroundLight}{RGB}{247, 247, 250}  % Soft off-white background

% --- TEXT AND STRUCTURE ---
\setbeamercolor{normal text}{fg=black!90, bg=backgroundLight}
\setbeamercolor{structure}{fg=primaryColor} % affects bullets, section titles, etc.

% --- TITLE AND PALETTES ---
\setbeamercolor{palette primary}{bg=primaryColor, fg=white}
\setbeamercolor{palette secondary}{bg=secondaryColor, fg=white}
\setbeamercolor{palette tertiary}{bg=accentColor!80, fg=black}
\setbeamercolor{title}{bg=secondaryColor, fg=white}
\setbeamercolor{background canvas}{bg=backgroundLight}

% --- HEADER AND FOOTER ---
\setbeamercolor{headline}{bg=primaryColor!95, fg=white}
\setbeamercolor{section in head/foot}{bg=secondaryColor, fg=white}
\setbeamercolor{subsection in head/foot}{bg=primaryColor!90, fg=white}
\setbeamercolor{author in head/foot}{bg=primaryColor!85, fg=white}
\setbeamercolor{title in head/foot}{bg=secondaryColor!90, fg=white}
\setbeamercolor{date in head/foot}{bg=primaryColor!80, fg=white}
\setbeamercolor{page number in head/foot}{bg=secondaryColor!85, fg=white}

% --- BLOCKS ---
\setbeamercolor{block title}{bg=secondaryColor!95, fg=white}
\setbeamercolor{block body}{bg=accentColor!15!white, fg=black!90}
\setbeamercolor{alerted text}{fg=primaryColor!85!black}
\setbeamercolor{example text}{fg=secondaryColor!70!black}

% --- OPTIONAL ---
\setbeamercolor{background canvas}{bg=backgroundLight!98!white}

%----------------------------------------------------------------------------------------
% BIBLIOGRAFIA
%----------------------------------------------------------------------------------------
\usepackage[
    style=abnt,
    giveninits=true,
    repeatfields=true,
    justify=raggedright,
    backref=true,
    backrefstyle=three+
]{biblatex}
\renewcommand*{\bibfont}{\small}
\addbibresource{bibliografia.bib}

%----------------------------------------------------------------------------------------
% INFORMAÇÕES DA APRESENTAÇÃO
%----------------------------------------------------------------------------------------
\title[Equações em Meios Porosos]{Estudo Numérico de Equações \\ em Meios Porosos}          % [Versão curta]{Versão completa}
\author[Henrique Casellato V. R. C.]{Henrique Casellato Vitorio Rodrigues da Costa}            % [Versão curta]{Nome completo}
\institute[FFCLRP-USP]{Departamento de Computação e Matemática \\ (DCM/FFCLRP-USP)}
\date[2025]{Dezembro / 2025}

%----------------------------------------------------------------------------------------
% INÍCIO DO DOCUMENTO
%----------------------------------------------------------------------------------------

\usepackage{ragged2e}
\justifying

\usepackage{graphicx}
\usepackage{animate}

\AtBeginSection[]
{
  \begin{frame}
    \frametitle{Estrutura da apresentação}
    \tableofcontents[currentsection] 
  \end{frame}
}

\begin{document}
% Slide de título com logo
\begin{frame}
  \begin{figure}[H]
    \centering
    \begin{subfigure}[c]{0.35\textwidth}
      \centering
      \includegraphics[width=\textwidth]{img/logo-dcm.png}
    \end{subfigure}
    \hspace{.05\textwidth}
    \begin{subfigure}[c]{0.35\textwidth}
      \centering
      \includegraphics[width=\textwidth]{img/logo-ffclrp.png}
    \end{subfigure}
  \end{figure}
  \titlepage
\end{frame}


% Sumário
\begin{frame}
    \frametitle{Estrutura da apresentação}
    \tableofcontents
\end{frame}

% Inclusão das seções
\section{Definições gerais} % Seções são adicionadas para organizar sua apresentação em blocos discretos, todas as seções e subseções são automaticamente exibidas no índice como uma visão geral da apresentação, mas NÃO são exibidas como slides separados.

%----------------------------------------------------------------------------------------

\begin{frame}
  \frametitle{O que é um meio poroso?}
  Um meio poroso é caracterizado por sua \textbf{porosidade} $\phi$, a fração entre o volume dos espaços vazios $V_p$ e o total $V_t$:
  \[
    \phi = \frac{V_p}{V_t},
  \]
  portanto sendo uma grandeza adimensional entre zero e um. 
  \begin{figure}[H]
    \centering
    % \begin{subfigure}[b]{0.32\textwidth}
    %   \centering
    %   \includegraphics[width=\textwidth]{img/Porousceramic.jpeg}
    %   \caption{Cerâmica \\
    %   {\tiny Fonte: Biofilter tech / Wikimedia Commons (CC BY-SA 3.0)}}
    % \end{subfigure}
    \begin{subfigure}[b]{0.32\textwidth}
      \centering
      \includegraphics[width=\textwidth]{img/ElTorcal0408.jpg}
      \caption{Calcário {\tiny (Torcal de Antequera)} \\
      {\tiny Fonte: Fernando Domínguez Cerejido / Wikimedia Commons (CC BY-SA 4.0)}}
    \end{subfigure}
    \hfill
    \begin{subfigure}[b]{0.32\textwidth}
      \centering
      \includegraphics[width=\textwidth]{img/Groundwater.png}
      \caption{Aquífero subterrâneo \\
      {\tiny Fonte: T.C. Winter et al., U.S. Geological Survey (Domínio Público)}}
    \end{subfigure}
    \hfill
    \begin{subfigure}[b]{0.32\textwidth}
      \centering
      \includegraphics[width=\textwidth]{img/Oilwells.jpg}
      \caption{Reservatório de petróleo \\
      {\tiny Fonte: G.H. Eldridge (c. 1905) / NOAA (Domínio Público)}}
    \end{subfigure}
    \caption{Exemplos de meios porosos.}
  \end{figure}
\end{frame}

%----------------------------------------------------------------------------------------

% \begin{frame}
%   \frametitle{Por que estudar meios porosos?}
%   Lorem ipsum dolor sit amet, consectetur adipiscing elit:
% \end{frame}

%----------------------------------------------------------------------------------------

\begin{frame}
	\frametitle{Compressibilidade e viscosidade}
  \textbf{Meios} podem ser considerados:
  \begin{itemize}
    \justifying
    \item \textbf{Incompressíveis}, se têm porosidade \textit{estática}; e
    \item \textbf{Compressíveis}, se têm porosidade \textit{dinâmica}.
  \end{itemize}

  Os \textbf{escoamentos}, por sua vez, podem ser classificados como:
  \begin{itemize}
    \justifying
    \item \textbf{Incompressíveis}: possui compressibilidade zero, o que implica uma massa específica constante independentemente da pressão;
    \item \textbf{Levemente compressíveis}: possui uma compressibilidade baixa e constante, com sua massa específica variando linearmente com a pressão; ou
    \item \textbf{Compressíveis}: possui compressibilidade alta e sua massa específica cresce com a pressão, tendendo a se estabilizar em altas pressões.
  \end{itemize}

  A \textbf{viscosidade} é a \textit{medida da resistência de um fluido ao próprio escoamento}. Os gases apresentam baixa viscosidade, e os fluidos mais densos (óleo) apresentam alta viscosidade.
\end{frame}

%----------------------------------------------------------------------------------------

\begin{frame}
  \frametitle{Permeabilidade do meio}
  A \textbf{permeabilidade} $K$ representa a \textit{habilidade} do meio poroso em \textit{transmitir um fluido através dos seus poros interconectados}, quando completamente saturado do mesmo. Quando somente dá um único fluido no meio poroso, esta propriedade recebe o nome de \textbf{permeabilidade absoluta}.

  A permeabilidade é dada por um tensor \textit{simétrico positivo definido}:
  \[
    K = 
    \begin{bmatrix}
      K_{xx} && K_{xy} && K_{xz} \\
      K_{yx} && K_{yy} && K_{yz} \\
      K_{zx} && K_{zy} && K_{zz}
    \end{bmatrix},
  \]
  caso ela seja \textbf{isotrópica}, $K$ é um escalar.

  Quando $K$ é espacialmente constante o meio se denomina \textbf{homogêneo} e caso contrário, \textbf{heterogêneo}.
\end{frame}

%----------------------------------------------------------------------------------------

% \begin{frame}
%   \frametitle{Viscosidade}
%   A \textbf{viscosidade} é a \textit{medida da resistência de um fluido ao próprio escoamento}. Os gases, que possuem moléculas distanciadas, apresentam baixa resistência ao escoamento e portanto, baixa viscosidade. Por outro lado, fluidos mais densos apresentam alta resistência ao escoamento, e consequentemente alta viscosidade.

%   \begin{itemize}
%     \justifying
%     \item No caso de \textbf{água} (um escoamento incompressível), vale notar que neste caso a viscosidade não varia quando a pressão aumenta;
%     \item Já no caso do \textbf{óleo}, a variação da viscosidade ocorre, pois o óleo pode transferir parte de sua massa para a forma de gás (escoamento compressível).
%   \end{itemize}

%   Vale ressaltar que é possível medir a variação da viscosidade em relação à pressão do reservatório, considerando o efeito da pressão nas massas específicas dos fluidos.
% \end{frame}

%----------------------------------------------------------------------------------------

\begin{frame}
  \frametitle{Lei de Darcy}
  Para introduzir a modelagem de escoamentos de fluidos, é preciso antes entender a \textbf{velocidade de Darcy}. Esta equação advém de estudos do fluxo físico da água em um filtro vertical de areia feitos por Henry Darcy em 1856.
  \begin{figure}[H]
    \centering
    \includegraphics[width=.5\textwidth]{img/experimento_Darcy.png}
    \caption{Experimento de Darcy, SOUZA e ROCHA (2022, p. 16).}
  \end{figure}
\end{frame}

%----------------------------------------------------------------------------------------

% \begin{frame}
%   \frametitle{Lei de Darcy}
%   Sua \textbf{altura hidráulica} $h$ (com $h_t$ no topo e $h_b$ na base) em relação a um ponto fixo $z$ dada por
%   \[
%     h = -\frac{p}{\rho g} + z,
%   \]
%   onde $g$ é a magnitude da aceleração da gravidade, $p$ é a pressão de água e $\rho$ é a massa específica da água. A taxa de fluxo $Q$ pode ser equacionada por
%   \[
%     \frac{Q}{A} = \mathbf{k}\frac{h_t - h_b}{L}\check{e},
%   \]
%   onde $\check{e}$ é o vetor direção do escoamento e $\mathbf{k}$ a condutividade hidráulica, que (nesse contexto) é dada por
%   \[
%     \mathbf{k} = \frac{\rho g K}{\mu},
%   \]
%   sendo $\mu$ é a viscosidade do fluido.
% \end{frame}

%----------------------------------------------------------------------------------------

\begin{frame}
  \frametitle{Lei de Darcy}
  Este fluxo representa o volume total do fluido pela área total por tempo, sendo também conhecido como \textbf{velocidade de Darcy} dada por:
  \begin{equation}\label{leiDeDarcy}
    \begin{aligned}
      u = \frac{Q}{A} = \mathbf{k}\frac{h_t - h_b}{L}\check{e} &= \frac{\rho g K}{\mu}\nabla h \\
      &= \frac{\rho g K}{\mu}\nabla \left(-\frac{p}{\rho g} + z\right) \\
      &= -\frac{K}{\mu} (\nabla p - \rho g \nabla z),
    \end{aligned}
  \end{equation}
  sendo $z$ a profundidade. Esta equação (\theequation)\ é conhecida como \textbf{Lei de Darcy} e representa a conservação de quantidade de movimento, na qual duas forças governam o fluxo: a pressão e a gravidade.
\end{frame}

%----------------------------------------------------------------------------------------
\section{Modelagem de escoamentos monofásicos}

%----------------------------------------------------------------------------------------

\begin{frame}{Modelagem de escoamentos monofásicos}
  Quando se aplica uma pressão (ou fluxo) em um domínio saturado por apenas \textbf{um fluido}, é induzido o que se chama de \textbf{escoamento monofásico}. A conservação de massa do fluido implica uma lei de balanço:

  \begin{equation}
      \frac{\partial}{\partial t}\int_V \phi\rho\ dx + \int_{\partial V}\rho (u \cdot n)\ ds = \int_V q\ dx.
  \end{equation}
  Aplicando-se o teorema da divergência, utilizando a Lei de Darcy e desenvolven-do-a, obtém-se a equação diferencial parcial de incógnitas $p$ e $\rho$:
  \begin{equation}
    \rho \phi c_t \frac{\partial p}{\partial t} - \nabla \cdot \left(\frac{\rho K}{\mu}(\nabla p - \rho g \nabla z)\right) = q,
  \end{equation}
  onde $c_t = c_f + c_r$ representa a compressibilidade total. Considerado um \textbf{escoamento incompressível}, a equação (\theequation)\ torna-se uma equação elíptica com coeficientes variáveis:
  \begin{equation}\label{eq246}
    - \nabla \cdot \left(\frac{\rho K}{\mu}(\nabla p - \rho g \nabla z)\right) = q.
  \end{equation}
\end{frame}

%----------------------------------------------------------------------------------------

\begin{frame}{Condições auxiliares}
  Além das equações que, de fato, modelam escoamentos monofásicos, faz-se necessário ainda condições de contorno. Estas condições podem ser classificadas principalmente em três tipos:

  \begin{itemize}
      \item \textbf{Dirichlet}, onde se dá pressão $p(x)$;
      \item \textbf{Neumann}, onde o fluxo $\nabla u \cdot n$ é dado; e
      \item \textbf{Robin} ou mista, onde é especificado $\alpha u + \beta (\nabla u \cdot n)$;
  \end{itemize}

  onde $u$ é uma função escalar, como por exemplo a temperatura ou pressão. Estas condições podem ser \textbf{homogêneas}, quando o contorno é igual a zero, ou \textbf{heterogêneas}, quando são iguais a uma função $g$.
\end{frame}

%----------------------------------------------------------------------------------------

\begin{frame}{Condições auxiliares}
  Para exemplificar os tipos de fronteiras nas bordas $\partial \Omega = \partial \Omega_p \cup \partial \Omega_u$, seguem as figuras:
    \begin{figure}[H]
    \centering
    \begin{subfigure}[b]{0.48\textwidth}
      \centering
      \begin{tikzpicture}[scale=1.6]
        \fill[fill=orange!40] (-1,-1) rectangle (1,1);
        \node[font=\Large] at (0,0) {$\Omega$};
        
        \node[left, above, font=\small] at (-1,1) {$\partial \Omega$};
        \fill[pattern=north east lines, pattern color=gray!80] (1,-1.1) rectangle (1.1,1.1);
        \draw[thick] (1,-1.1) -- (1,1.1);
        \draw[black] (-1,-1) rectangle (1,1);

        \node[rotate = 90, right, anchor=center] at (1.3,0) {$p(x,y) = 0$};
      \end{tikzpicture}
      \caption{Contorno Dirichlet homogêneo.}
    \end{subfigure}
    \hfill
    \begin{subfigure}[b]{0.48\textwidth}
      \centering
      \begin{tikzpicture}[scale=1.6]
        \fill[fill=orange!40] (-1,-1) rectangle (1,1);
        \node[font=\Large] at (0,0) {$\Omega$};
        
        \node[left, font=\small] at (-1,1) {$\partial \Omega$};
        \fill[pattern=north east lines, pattern color=gray!80] (-1.1,-1) rectangle (1.1,-1.1);
        \draw[thick] (-1.1,-1) -- (1.1,-1);
        \draw[black] (-1,-1) rectangle (1,1);

        \foreach \y in {-0.8,-0.4,0.0,0.4,0.8}
        {
          \draw[->, thick, black!60] (\y,-.9) -- (\y,-1.1);
        }

        \node[below, anchor=center] at (0,-1.5) {$-K\frac{\partial p}{\partial n} = g(x,y)$};
      \end{tikzpicture}
      \caption{Contorno Neumann não homogêneo.}
    \end{subfigure}
    \caption{Exemplificação de tipos de contorno.}
  \end{figure}
\end{frame}

%----------------------------------------------------------------------------------------

\begin{frame}{Meios homogêneos e heterogêneos}
  Para classificar meios como homogêneos ou heterogêneos, considera-se uma simplificação de \eqref{eq246}\ para o escoamento monofásico, com as hipóteses do escoamento ser \textit{incompressível, isotérmico e sem efeito gravitacional}, da forma que:
  \begin{equation}\label{escoamento_base}
  \left\{
    \begin{aligned}
      \nabla \cdot u &= \frac{q}{\rho} && \text{em } \Omega \\
      u &= -\frac{K}{\mu} \nabla p && \text{em } \Omega \\
      p &= p_b && \text{em } \partial\Omega_p \\
      u \cdot n &= u_b && \text{em } \partial\Omega_u
    \end{aligned}
  \right.
  ,
  \end{equation}
  onde a viscosidade $\mu$ e massa específica $\rho$ são uniformes no caso incompressível e consideradas unitárias aqui. Também, onde $\partial \Omega = \partial \Omega_p \cup \partial \Omega_u$ é a fronteira do domínio $\Omega \subset \mathbb{R}^d$ com $d = \{1,2,3\}$.
\end{frame}

%----------------------------------------------------------------------------------------

\begin{frame}{Meios homogêneos e heterogêneos}
  Portanto, o meio pode ser classificado, principalmente entre:
  \begin{itemize}
    \justifying
    \item Um meio dito \textbf{homogêneo}, governado pelo sistema de escoamentos mono-fásicos (\theequation), ocorre quando a permeabilidade absoluta $K$ é constante e uniforme, logo não dependendo de $x$; ou
    \item Quando o meio é \textbf{heterogêneo}, a permeabilidade absoluta $K$ varia com $x$ e então o escoamento tende a passar pelas regiões de alta permeabilidade e evitar as de baixa permeabilidade.
  \end{itemize}
  Também, é possível usar um modelo do projeto \textit{SPE10} fornecido pela \textit{Sociedade de Engenheiros de Petróleo}, utilizado como referência em simulações de reservatórios de petróleo.
\end{frame}

%----------------------------------------------------------------------------------------

\begin{frame}{Método de volumes finitos}
  O \textbf{método de volumes finitos} é um método de discretização útil para simulações numéricas de leis de conservação de vários tipos: \textit{elípticas, hiperbólicas ou parabólicas}, por exemplo. Este método tem alguns pontos importantes: 
  \begin{itemize}
    \justifying
    \item Pode ser usado em \textbf{geometrias arbitrárias};
    \item Pode ser usado em \textbf{malhas estruturadas ou não}; e
    \item \textbf{Conserva localmente os fluxos numéricos}, o que o faz particularmente interessante para problemas de mecânica dos fluidos.
  \end{itemize}
  Isso pode ser obtido, pois é baseado em uma abordagem "balanceada":
  \begin{itemize}
    \justifying
    \item Um {balanço local} é escrito em cada célula de discretização, que é frequentemente chamada de \textit{volume de controle};
    \item Pela fórmula de divergência, uma {formulação da integral} dos fluxos sobre a fronteira do volume de controle é obtida.
  \end{itemize}
  Os fluxos nas fronteiras são discretizados com respeito aos discretos "desconhecidos".
\end{frame}

%----------------------------------------------------------------------------------------

\begin{frame}{Malha centrada em células}
  Durante o restante desse trabalho, será usada uma discretização \textbf{centrada em células}, onde é considerado que o \textit{valor aproximado da função}, ou da média dos valores da célula, \textit{está justamente no centro dela}.

  Para visualizar melhor, toma-se o intervalo $I = [0,1]$ com três subintervalos de espaçamento uniforme:
  \begin{figure}[H]
    \centering
    \begin{tikzpicture}[scale=1.5]
      % Eixo x
      \draw[black, dashed, <->] (-.5,0) -- (6.5,0);
      \draw[black,thick] (0,0) -- (6,0);
      
      % Linhas da malha
      \filldraw[black] (1,0) circle (1pt);
      \filldraw[black] (3,0) circle (1pt);
      \filldraw[black] (5,0) circle (1pt);

      \node[above, font=\large] at (1,0.1) {$a_{1}$};
      \node[above, font=\large] at (3,0.1) {$a_{2}$};
      \node[above, font=\large] at (5,0.1) {$a_{3}$};

      \draw[black]     (0,-0.1) -- (0,0.1);
      \draw[black]     (2,-0.1) -- (2,0.1);
      \draw[black]     (4,-0.1) -- (4,0.1);
      \draw[black]     (6,-0.1) -- (6,0.1);

      \node[below, font=\large] at (0,-0.1) {$a_{\frac{1}{2}} = 0$};
      \node[below, font=\large] at (2,-0.1) {$a_{\frac{3}{2}}$};
      \node[below, font=\large] at (4,-0.1) {$a_{\frac{5}{2}}$};
      \node[below, font=\large] at (6,-0.1) {$a_{\frac{7}{2}} = 1$};

      \node[right, font=\large] at (6.5,0) {$x$};

    \end{tikzpicture}
    \caption{Intervalo $I$ com discretização no centro da célula e contorno de Neumann.}
    \label{fig:viz_CCG_N}
  \end{figure}
  Os valores dos nós $a_i$ são vistos como os \textbf{centros} das células, então $a_{i-1/2}$ e $a_{i+1/2}$ seriam suas faces à esquerda e à direita. No caso de \textit{contornos de Neumann} na primeira célula, durante a integração de $[a_{\frac{1}{2}},a_{\frac{3}{2}}]$, o fluxo vindo de $a_{\frac{1}{2}}$ seria a \textit{imposição de fluxo da interface esquerda} do contorno de Neumann. 
\end{frame}

%----------------------------------------------------------------------------------------

\begin{frame}{Malha centrada em células}
  Geralmente, nos contornos de Dirichlet, inclui-se uma \textbf{célula fantasma} como na figura abaixo:
  \begin{figure}[H]
    \centering
    \begin{tikzpicture}[scale=1]
      % Eixo x
      \draw[black, dashed, <->] (-2.5,0) -- (7.5,0);
      \draw[black]              (-2,  0) -- (0,  0);
      \draw[black,thick]        ( 0,  0) -- (6,  0);
      % Linhas da malha
      \filldraw[black] (-1,0) circle (1pt);
      
      \filldraw[black] (1,0) circle (1pt);
      \filldraw[black] (3,0) circle (1pt);
      \filldraw[black] (5,0) circle (1pt);

      \node[above, font=\large] at (-1,0.15) {$a_{0}$};

      \node[above, font=\large] at (1,0.1) {$a_{1}$};
      \node[above, font=\large] at (3,0.1) {$a_{2}$};
      \node[above, font=\large] at (5,0.1) {$a_{3}$};

      \draw[black] (-2,-0.1) -- (-2,0.1);

      \draw[black] ( 0,-0.15) -- ( 0,0.15);
      \draw[black] ( 2,-0.1) -- ( 2,0.1);
      \draw[black] ( 4,-0.1) -- ( 4,0.1);
      \draw[black] ( 6,-0.15) -- ( 6,0.15);

      \node[below, font=\large] at (-2,-0.1) {$a_{-\frac{1}{2}}$};

      \node[below, font=\large] at (0,-0.15)  {$a_{\frac{1}{2}}$};
      \node[below, font=\large] at (2,-0.1)  {$a_{\frac{3}{2}}$};
      \node[below, font=\large] at (4,-0.1)  {$a_{\frac{5}{2}}$};
      \node[below, font=\large] at (6,-0.15)  {$a_{\frac{7}{2}}$};

      \node[right, font=\large] at (7.5,0) {$x$};

    \end{tikzpicture}
    \caption{Intervalo $I$ com discretização no centro da célula, contorno de Dirichlet à esquerda usando célula fantasma.}
    \label{fig:viz_CCG_D}
  \end{figure}

  No exemplo do intervalo $I$ descrito, a célula fantasma seria a célula com centro em $a_0$, então se utilizando desse valor na integração em $[a_{\frac{1}{2}},a_{\frac{3}{2}}]$. Por simplicidade, o contorno à direita se manteve de Neumann.

  Também, é possível usar meia célula, como em:
  \begin{figure}[H]
    \centering
    \begin{tikzpicture}[scale=1]
      % Eixo x
      \draw[black, dashed, <->] (-2.0,0) -- (7.5,0);
      \draw[black]              (-1,  0) -- (0,  0);
      \draw[black,thick]        ( 0,  0) -- (6,  0);
      % Linhas da malha
      \filldraw[black] (-1,0) circle (1pt);
      
      \filldraw[black] (1,0) circle (1pt);
      \filldraw[black] (3,0) circle (1pt);
      \filldraw[black] (5,0) circle (1pt);

      \node[above, font=\large] at (-1,0.15) {$a_{0} = 0$};

      \node[above, font=\large] at (1,0.1) {$a_{1}$};
      \node[above, font=\large] at (3,0.1) {$a_{2}$};
      \node[above, font=\large] at (5,0.1) {$a_{3}$};

      \draw[black] (-1,-0.1) -- (-1,0.1);

      \draw[black] ( 0,-0.15) -- ( 0,0.15);
      \draw[black] ( 2,-0.1) -- ( 2,0.1);
      \draw[black] ( 4,-0.1) -- ( 4,0.1);
      \draw[black] ( 6,-0.15) -- ( 6,0.15);

      \node[below, font=\large] at (0,-0.15) {$a_{\frac{1}{2}}$};
      \node[below, font=\large] at (2,-0.1 ) {$a_{\frac{3}{2}}$};
      \node[below, font=\large] at (4,-0.1 ) {$a_{\frac{5}{2}}$};
      \node[below, font=\large] at (6,-0.15) {$a_{\frac{7}{2}} = 1$};

      \node[right, font=\large] at (7.5,0) {$x$};

    \end{tikzpicture}
    \caption{Intervalo $I$ com discretização no centro da célula, contorno de Dirichlet à esquerda usando meia célula.}
    \label{fig:viz_CCG_Dh}
  \end{figure}
\end{frame}

%----------------------------------------------------------------------------------------
\section{Método de volumes finitos para equações elípticas} % Seções são adicionadas para organizar sua apresentação em blocos discretos, todas as seções e subseções são automaticamente exibidas no índice como uma visão geral da apresentação, mas NÃO são exibidas como slides separados.

%----------------------------------------------------------------------------------------

\begin{frame}{Método de volumes finitos para equações elípticas}
  O objetivo desta seção é aproximar numericamente a solução do sistema \eqref{eq246} com simplificações:
  \begin{equation}\label{escoamento_base}
  \left\{
    \begin{aligned}
      - \nabla \cdot (K \nabla p) &= q && \text{em } \Omega \\
      p &= p_b && \text{em } \partial\Omega_p \\
      (-K \nabla p) \cdot n &= u_b && \text{em } \partial\Omega_u
    \end{aligned}
  \right.,
  \end{equation}
  no caso de um escoamento monofásico, com as hipóteses de ser incompressível, isotérmico e sem efeito gravitacional, de viscosidade $\mu$ e massa específica $\rho$ constantes e unitárias.

  Ademais, tem pressão relacionada a velocidade de Darcy
  \begin{equation}
    u = - K \nabla p.
  \end{equation}
\end{frame}

%----------------------------------------------------------------------------------------

% \begin{frame}{Método de volumes finitos para equações elípticas}
%   Para obter uma discretização de volumes finitos, integra-se a primeira equação de \eqref{escoamento_base} sobre um volume de controle genérico $V_k$:
%   \begin{equation}
%     - \int_{V_k}\nabla \cdot (K \nabla p)\ dx = \int_{V_k} q\ dx
%   \end{equation}
%   e com o teorema da divergência,
%   \begin{equation}\label{eq:eq44}
%     - \int_{\partial V_k}(K \nabla p) \cdot n_k\ ds = \int_{V_k} q\ dx,
%   \end{equation}
%   em que $n_k$ é o vetor normal à $V_k$.
% \end{frame}

%----------------------------------------------------------------------------------------

\begin{frame}{Caso unidimensional}
  Considerando um domínio unidimensional $\Omega = [a,b]$ onde a primeira equação de \eqref{escoamento_base} é escrita como
  \begin{equation}\label{eq45}
    \frac{d}{dx}\left(K\frac{dp}{dx}\right) = q.
  \end{equation}
  Na discretização por volumes finitos, têm-se uma partição do domínio em $N$ intervalos $V_i = [x_{i-1/2},x_{i+1/2}]$ e integrando a equação em qualquer $V_i$,
  \begin{equation}\label{eq_1DfullequationVF}
    -\int_{x_{i-1/2}}^{x_{i+1/2}}q\ dx = \int_{x_{i-1/2}}^{x_{i+1/2}}\frac{d}{dx}\left(K\frac{dp}{dx}\right)\ dx = \left.K\frac{dp}{dx}\right|_{x_{i+1/2}} - \left.K\frac{dp}{dx}\right|_{x_{i-1/2}}.
  \end{equation}
  Considerando que $\Delta x_{i+1} = \Delta x_{i} = \Delta x$,
  \begin{equation}
    \left.K\frac{dp}{dx}\right|_{x_{i+1/2}} = \frac{2K_iK_{i+1}}{K_i + K_{i+1}}\left(\frac{p_{i+1} - p_{i}}{\Delta x}\right),
  \end{equation}
  a equação \eqref{eq_1DfullequationVF} em cada volume $V_i$ é dada por
  \begin{equation}\label{esquemaNum420}
    -\frac{1}{\Delta x^2}\left(K_{i+1/2}(p_{i+1} - p_{i}) - K_{i-1/2}(p_{i} - p_{i-1})\right) = q_i.
  \end{equation}
\end{frame}

%----------------------------------------------------------------------------------------

\begin{frame}{Caso unidimensional: Implementação}
  Para implementar o método de volumes finitos, considera-se o esquema numérico \eqref{esquemaNum420} escrito em forma matricial, com $N$ células
  \begin{equation}\label{forma_matricial_1d}
    Aw = d,
  \end{equation}
  com $w$ e $d$ representando os vetores de valores aproximados para a pressão e os de injeção $q$. Por fim, $A$ é uma matriz tridiagonal definida por
  \begin{equation}
    (Aw)_i = \frac{1}{\Delta x^2}\left(-K_{i-\frac{1}{2}}p_{i-1}+\left(K_{i-\frac{1}{2}}+K_{i+\frac{1}{2}}\right)p_i-K_{i+\frac{1}{2}}p_{i+1}\right),
  \end{equation}
  para $i = 2,...,N-1$. No contexto de uma malha centrada em células, as linhas $i=1$ e $i=N$ da matriz são usadas para impor condições de contornos que, neste seguinte exemplo explicativo, são de Neumann à esquerda e de Dirichlet à direita.
\end{frame}

%----------------------------------------------------------------------------------------

% \begin{frame}{Caso unidimensional: Implementação}
%   \textbf{Contorno de Neumann}, quando $i = 1$, a condição de contorno à esquerda estaria na célula $x_{i-1/2} = x_{1/2}$. Então, com
%   \[
%     - \left.K\frac{dp}{dx}\right|_{x_{\frac{1}{2}}} = u_b,
%   \]
%   para a primeira equação do sistema \eqref{forma_matricial_1d},
%   \[
%   \begin{aligned}
%     - \left(\left.K\frac{dp}{dx}\right|_{x_{\frac{3}{2}}} - \left.K\frac{dp}{dx}\right|_{x_{\frac{1}{2}}}\right) &= \Delta x q_1 \\
%     \frac{2K_1K_{2}}{K_1 + K_{2}}\left(\frac{p_2 - p_1}{\Delta x}\right) &= \Delta xq_1 + u_b
%   \end{aligned}
%   \]
%   ou ainda:
%   \[
%     \frac{1}{\Delta x^2}\left(-K_{\frac{3}{2}}(p_2-p_1)\right) = q_1 + \frac{u_b}{\Delta x}
%   \]
%   e poderia ser feito na extremidade oposta de forma similar, caso necessário.
% \end{frame}

%----------------------------------------------------------------------------------------

\begin{frame}{Caso unidimensional: Implementação}
  \textbf{Contorno de Dirichlet}, quando $i = N$, a face $x_{N+1/2}$, tem pressão imposta
  \[
    \left.p\right|_{x_{N+1/2}} = p_b
  \]
  e então a última equação do sistema \eqref{forma_matricial_1d} seria, portanto,
  \begin{equation}\label{eq428}
    - \left(\left.K\frac{dp}{dx}\right|_{x_{N+\frac{1}{2}}} - \left.K\frac{dp}{dx}\right|_{x_{N-\frac{1}{2}}}\right) = \Delta x q_N.
  \end{equation}
  Para este exemplo de contorno Dirichlet, será usada uma discretização com meio volume de controle:
  \[
    \left.K\frac{dp}{dx}\right|_{x_{N+\frac{1}{2}}} = K_N\left(\frac{p_b - p_N}{\Delta x/2}\right) = 2K_N\left(\frac{p_b - p_N}{\Delta x}\right)
  \]
  Dessa forma, em \eqref{eq428}:
  \[
    \frac{1}{\Delta x^2}\left(2K_{N}p_N + K_{N-\frac{1}{2}}(p_N-p_{N-1})\right) = q_1 + \frac{2K_{N}p_b}{\Delta x^2}
  \]
  e poderia ser feito na extremidade oposta de forma similar, caso necessário.
\end{frame}

%----------------------------------------------------------------------------------------

\begin{frame}{Caso unidimensional: Implementação}
  Portanto, a matriz $A$ do sistema, considerando as duas condições de contorno (de Neumann pela esquerda e de Dirichlet pela direita) é dada por
  \[
    A = \frac{1}{\Delta x^2}
    \begin{bmatrix}
      K_{\frac{3}{2}}  & -K_{\frac{3}{2}} & & & &  \\
      -K_{\frac{3}{2}} & \left(K_{\frac{3}{2}} + K_{\frac{5}{2}}\right) & -K_{\frac{5}{2}}   & & & \\
      & \ddots & \ddots & \ddots & & \\
      & & -K_{N-\frac{3}{2}} & \left(K_{N-\frac{3}{2}} + K_{N-\frac{1}{2}}\right) & -K_{N-\frac{1}{2}} \\
      & & & -K_{N-\frac{1}{2}} & \left(K_{N-\frac{1}{2}} + 2 K_{N}\right)
    \end{bmatrix}
  \]
  e
  \[
    d^t = 
    \left(
      q_1 + \frac{u_b}{\Delta x},\ q_2,\ \cdots,\ q_{N-1},\ q_N + \frac{2K_{N}p_b}{\Delta x^2}
    \right).
  \]
\end{frame}

%----------------------------------------------------------------------------------------

\begin{frame}{Caso unidimensional: Implementação}
  Foram realizados testes com os três algoritmos de fatoração: $\mathbf{LDL^T}$, de \textbf{Crout} e de \textbf{Cholesky}. Pode-se ver que o método $\mathbf{LDL^T}$ performou melhor que a fatoração de Crout (pensando no problema do exemplo desta seção).
  \begin{table}[h]
    \centering
    \begin{tabular}{|c|c|c|c|}
      \hline
      \textbf{Subintervalos} & $\mathbf{LDL^t}$ & \textbf{Crout} & \textbf{Cholesky} \\
      \hline\hline
      $10^2$ & $0.18247 \times 10^{-4}\ s$ & $0.18758 \times 10^{-4}\ s$ & $0.20147 \times 10^{-4}\ s$ \\
      $10^3$ & $0.86234 \times 10^{-4}\ s$ & $0.95469 \times 10^{-4}\ s$ & $0.98242 \times 10^{-4}\ s$ \\
      $10^4$ & $0.82834 \times 10^{-3}\ s$ & $0.92428 \times 10^{-3}\ s$ & $0.94293 \times 10^{-3}\ s$ \\
      $10^5$ & $0.72082 \times 10^{-2}\ s$ & $0.79367 \times 10^{-2}\ s$ & $0.83214 \times 10^{-2}\ s$ \\
      \hline
    \end{tabular}
    \caption{\justifying Tempos médios de métodos de fatoração diferentes.}
  \end{table}
  Os três algoritmos possuem a mesma ordem de erro de truncamento $O(h^2)$.
\end{frame}

%----------------------------------------------------------------------------------------

\begin{frame}{Caso unidimensional: Exemplo contínuo}
\begin{exemplo}
  \justifying
  Dado um problema de valor de contorno unidimensional
  \begin{equation*}
    \left\{
      \begin{aligned}
      -\frac{d}{dx}\left(K\frac{dp}{dx}\right) &= -25\cos(25x) && \text{em $\Omega = [0,1]$} \\
      p &= x && \text{sobre $\partial\Omega$}
      \end{aligned}
    \right.,
  \end{equation*}
  onde a permeabilidade absoluta do meio é $K(x) = 2 + \sin(25x)$ e solução exata $p(x) = x$. As fatorações produziram resultados próximos à solução exata, com erro de truncamento $\epsilon \approx 0.355\times 10^{-2}$, e gráficos:
\end{exemplo}
\end{frame}

%----------------------------------------------------------------------------------------

\begin{frame}{Caso unidimensional: Exemplo contínuo}
\begin{exemplo}
  \begin{figure}[H]
    \centering
    \begin{subfigure}[b]{0.45\textwidth}
      \centering
      \includegraphics[width=\textwidth]{img/desenvolvimento_CF.jpeg}
      \label{fig:img1}
    \end{subfigure}
    %\hfill % Adds horizontal space
    \begin{subfigure}[b]{0.45\textwidth}
      \centering
      \includegraphics[width=\textwidth]{img/desenvolvimento_LDL.jpeg}
      \label{fig:img2}
    \end{subfigure}
    \caption{Solução por diferentes métodos de fatoração (Crout e $LDL^t$)}
    \label{comparacaoEx1CFLDL}
  \end{figure}
\end{exemplo}
\end{frame}

%----------------------------------------------------------------------------------------

\begin{frame}{Caso bidimensional}
  A generalização do método de volumes finitos para uma dimensão maior segue um procedimento análogo ao da seção anterior. Ao final do desenvolvimento, a forma discreta para um problema bidimensional é
  \begin{equation}\label{eq436}
    \begin{aligned}
    q_{i,j} = &- \frac{1}{\Delta y^2} K_{i,j+\frac{1}{2}}p_{i,j+1} - \frac{1}{\Delta y^2} K_{i,j-\frac{1}{2}}p_{i,j-1} \\
    &- \frac{1}{\Delta x^2} K_{i+\frac{1}{2},j}p_{i+1,j} - \frac{1}{\Delta x^2} K_{i-\frac{1}{2},j}p_{i-1,j} \\
    &+ \left(\frac{1}{\Delta y^2} K_{i,j+\frac{1}{2}} + \frac{1}{\Delta y^2} K_{i,j-\frac{1}{2}} + \frac{1}{\Delta x^2} K_{i+\frac{1}{2},j} + \frac{1}{\Delta x^2} K_{i-\frac{1}{2},j}\right)p_{i,j},
    \end{aligned}
  \end{equation}
  onde se usam as médias harmônicas em $K_{i,j\pm 1/2}$ e $K_{i\pm 1/2,j}$.

  Após a pressão ser calculada, o campo de velocidades pode ser calculado usando a mesma estratégia de aproximação dos fluxos nas integrais em $\partial V_{i,j}$:
    \begin{equation*}
      \begin{aligned}
        u_L \simeq -K_{i+\frac{1}{2},j}\frac{p_{i+1,j} - p_{i,j}}{\Delta x}, && u_O \simeq -K_{i-\frac{1}{2},j}\frac{p_{i,j} - p_{i-1,j}}{\Delta x} \\
        v_N \simeq -K_{i,j+\frac{1}{2}}\frac{p_{i,j+1} - p_{i,j}}{\Delta y}, && v_S \simeq -K_{i,j-\frac{1}{2}}\frac{p_{i,j} - p_{i,j-1}}{\Delta y}.
      \end{aligned}
    \end{equation*}
\end{frame}

%----------------------------------------------------------------------------------------

% \begin{frame}{Caso bidimensional}
%   Após a pressão ser calculada, o campo de velocidades pode ser calculado usando a mesma estratégia de aproximação dos fluxos nas integrais em $\partial V_{i,j}$. Pela definição da velocidade de Darcy \eqref{leiDeDarcy}:
%     \[
%       \mathsf{u} = 
%       \begin{bmatrix}
%         u \\ v
%       \end{bmatrix} =
%       \begin{bmatrix}
%         -K\partial_xp \\ -K \partial_yp
%       \end{bmatrix}.
%     \]
%     Usando-se da discretização dos fluxos discutida anteriormente, mantendo-se a aproximação conservativa de volumes finitos, consegue-se:
%     \begin{equation*}
%       \begin{aligned}
%         u_L \simeq -K_{i+\frac{1}{2},j}\frac{p_{i+1,j} - p_{i,j}}{\Delta x}, && u_O \simeq -K_{i-\frac{1}{2},j}\frac{p_{i,j} - p_{i-1,j}}{\Delta x} \\
%         v_N \simeq -K_{i,j+\frac{1}{2}}\frac{p_{i,j+1} - p_{i,j}}{\Delta y}, && v_S \simeq -K_{i,j-\frac{1}{2}}\frac{p_{i,j} - p_{i,j-1}}{\Delta y}.
%       \end{aligned}
%     \end{equation*}
%     Para fins de visualização, o campo vetorial pode ser calculado no centro das células, fornecendo um campo discreto mais conveniente para a maioria das situações. Isso pode ser calculado por médias simples, da forma
%     \begin{equation}
%       \begin{aligned}
%         u_{i,j} = \frac{u_L + u_O}{2} && \text{e} && v_{i,j} = \frac{v_N + v_S}{2}.
%       \end{aligned}
%     \end{equation}
% \end{frame}

%----------------------------------------------------------------------------------------

% \begin{frame}{Caso bidimensional: Implementação}
%   Em contornos de tipo \textbf{Neumann}, há a imposição de um fluxo na fronteira:
%   \[
%     \begin{aligned}
%       -K\frac{\partial p}{\partial n} = -(K\nabla p) \cdot n = g(x,y) && \text{em $\zeta \subset \partial V_{i,j}$,}
%     \end{aligned}
%   \]
%   onde $g$ é uma função conhecida. Se $g$ é nula, têm-se um contorno do tipo \textbf{homogênea} e tomando o exemplo da face leste:
%   \[
%     \int_{L}(K\nabla p) \cdot n_L\ ds = 0.
%   \]
%   Portanto, sua contribuição na equação \eqref{integracaoDasFaces}\ será eliminada. Caso $g$ não seja nula, têm-se uma condição de contorno do tipo \textbf{não homogênea}:
%   \[
%     \int_{L}(K\nabla p) \cdot n_L\ ds = \int_{y_{j-1/2}}^{y_{j+1/2}}(K\nabla p) \cdot n_L\ dy = -\int_{y_{j-1/2}}^{y_{j+1/2}}g(x_{\frac{1}{2}},y)\ dy.
%   \]
% \end{frame}

%----------------------------------------------------------------------------------------

% \begin{frame}{Caso bidimensional: Implementação}
%   Desse modo, seria preciso integrar numericamente $g$, por exemplo, pela regra do trapézio:
%   \[
%     G_j = \int_{y_{j-1/2}}^{y_{j+1/2}}g(x_{\frac{1}{2}},y)\ dy \simeq \frac{\Delta y}{2}\left(g(x_{\frac{1}{2}},y_{i-\frac{1}{2}}) + g(x_{\frac{1}{2}},y_{i+\frac{1}{2}})\right),
%   \]
%   da forma que a equação \eqref{eq436}\ será modificada (no exemplo de quando $i = M$) para
%   \[
%     \begin{aligned}
%     q_{M,j}  - \frac{G_j}{\Delta x \Delta y} =
%     &- \frac{1}{\Delta y^2} K_{M,j+\frac{1}{2}}p_{M,j+1} - \frac{1}{\Delta y^2} K_{M,j-\frac{1}{2}}p_{M,j-1} - \frac{1}{\Delta x^2} K_{M-\frac{1}{2},j}p_{M-1,j} \\
%     &+ \left(\frac{1}{\Delta y^2} K_{M,j+\frac{1}{2}} + \frac{1}{\Delta y^2} K_{M,j-\frac{1}{2}} + \frac{1}{\Delta x^2} K_{M-\frac{1}{2},j}\right)p_{M,j}.
%     \end{aligned}
%   \]
% \end{frame}

%----------------------------------------------------------------------------------------

% \begin{frame}{Caso bidimensional: Implementação}
%   Em contornos de tipo \textbf{Dirichlet}, impõe-se uma pressão em alguma fronteira:
%   \[
%     \begin{aligned}
%     p = g(x,y) && \text{em $\zeta \subset \partial V_{i,j}$,}
%     \end{aligned}
%   \]
%   onde $g$ é uma função. Por exemplo, a face sul do volume $V_{i,1}$ está em $\zeta$, então a contribuição da integral deve ser recalculada para:
%   \[
%     \begin{aligned}
%         \int_{S}(K\nabla p)\cdot n_S\ ds 
%         &= \int_{x_{i-1/2}}^{x_{i+1/2}}-K\frac{\partial p}{\partial y}(x,y_{\frac{1}{2}})\ dx \\
%         &\simeq \int_{x_{i-1/2}}^{x_{i+1/2}}-K\left(\frac{p_{i,1} - g\left(x,y_{\frac{1}{2}}\right)}{y_1 - y_{\frac{1}{2}}}\right)\ dx \\
%         &= -2\frac{\Delta x}{\Delta y}K_{i,\frac{1}{2}}p_{i,1} + \frac{2}{\Delta y}K_{i,\frac{1}{2}}\int_{x_{i-1/2}}^{x_{i+1/2}}g(x,y_{\frac{1}{2}})\ dx.
%     \end{aligned}
%   \]
% \end{frame}

%----------------------------------------------------------------------------------------

% \begin{frame}{Caso bidimensional: Implementação}
%   E novamente, integra-se $g$ em $G_i$:
%   \[
%     G_i = \int_{x_{j-1/2}}^{x_{j+1/2}}g(x,y_{\frac{1}{2}})\ dx \simeq \frac{\Delta x}{2}\left(g(x_{i-\frac{1}{2}},y_{\frac{1}{2}}) + g(x_{i+\frac{1}{2}},y_{\frac{1}{2}})\right),
%   \]
%   e portanto, a equação \eqref{eq436}\ será modificada (no exemplo de quando $j = 1$) para 
%   \[
%     \begin{aligned}
%     q_{M,j}  - \frac{2}{\Delta y^2 \Delta x}G_iK_{i,\frac{1}{2}} =
%     &- \frac{1}{\Delta y^2} K_{i,\frac{3}{2}}p_{i,2} - \frac{1}{\Delta x^2} K_{i+\frac{1}{2},1}p_{i+1,1} - \frac{1}{\Delta x^2} K_{i-\frac{1}{2},1}p_{i-1,j} \\
%     &+ \left(\frac{1}{\Delta y^2} K_{i,\frac{3}{2}} + \frac{2}{\Delta y^2} K_{i,\frac{1}{2}} + \frac{1}{\Delta x^2} K_{i+\frac{1}{2},1} + \frac{1}{\Delta x^2} K_{i-\frac{1}{2},1}\right)p_{i,1},
%     \end{aligned}
%   \]
%   com $K_{i,\frac{1}{2}} = K_{i,1}$. Da mesma forma como no contorno de Neumann, quando $g \neq 0$, a condição de contorno de Dirichlet é do tipo \textbf{não homogêneo} e, caso contrário, \textbf{homogêneo}.
% \end{frame}

%----------------------------------------------------------------------------------------

\begin{frame}{Caso bidimensional: Implementação}
  Considerando uma discretização $M \times N$ células computacionais, o esquema \eqref{eq436} pode ser escrito na forma matricial
  \[
    Aw = d
  \]
  com $w = (p_1,...,p_{MN})^t$ e $b = (q_1,...,q_{MN})^t$. Cada linha da matriz $A$ está relacionada com uma célula $(i,j)$ e leva em consideração as contribuições de seus quatro vizinhos. Por exemplo, em uma malha $3 \times 3$, as células podem ser ordenadas \textit{por linhas} da esquerda para direita, de cima para baixo:

  \begin{figure}[H]
  \centering
  \begin{tikzpicture}[scale=1.5]
    \fill[fill=orange!20] (-0.4,1.2) rectangle (0.4,-1.2);
    \fill[fill=orange!20] (-1.2,0.4) rectangle (1.2,-0.4);
    \fill[fill=orange!40] (-0.4,0.4) rectangle (0.4,-0.4);

    \draw[black] (-1.2,-1.2) rectangle (1.2,1.2); % Desenha o quadrado

    % Linhas verticais (dividindo em 3 colunas)
    \draw[black] (-0.4,-1.2) -- (-0.4,1.2); % 1ª linha vertical
    \draw[black] ( 0.4,-1.2) -- ( 0.4,1.2);   % 2ª linha vertical

    % Linhas horizontais (dividindo em 3 linhas)
    \draw[black] (-1.2,-0.4) -- (1.2,-0.4); % 1ª linha horizontal
    \draw[black] (-1.2,0.4) -- (1.2,0.4);   % 2ª linha horizontal
    
    \node[below, font=\tiny, blue] at (-0.8,-0.8) {$1$};
    \node[below, font=\tiny, blue] at ( 0.0,-0.8) {$2$};
    \node[below, font=\tiny, blue] at ( 0.8,-0.8) {$3$};
    \node[below, font=\tiny, blue] at (-0.8, 0.0) {$4$};
    \node[below, font=\tiny, blue] at ( 0.0, 0.0) {$5$};
    \node[below, font=\tiny, blue] at ( 0.8, 0.0) {$6$};
    \node[below, font=\tiny, blue] at (-0.8, 0.8) {$7$};
    \node[below, font=\tiny, blue] at ( 0.0, 0.8) {$8$};
    \node[below, font=\tiny, blue] at ( 0.8, 0.8) {$9$};

    \node[above, font=\tiny, black] at (-0.8,-0.8) {$i-1,j-1$};
    \node[above, font=\tiny, black] at ( 0.0,-0.8) {$i  ,j  $};
    \node[above, font=\tiny, black] at ( 0.8,-0.8) {$i+1,j+1$};
    \node[above, font=\tiny, black] at (-0.8, 0.0) {$i-1,j-1$};
    \node[above, font=\tiny, black] at ( 0.0, 0.0) {$i  ,j  $};
    \node[above, font=\tiny, black] at ( 0.8, 0.0) {$i+1,j+1$};
    \node[above, font=\tiny, black] at (-0.8, 0.8) {$i-1,j-1$};
    \node[above, font=\tiny, black] at ( 0.0, 0.8) {$i  ,j  $};
    \node[above, font=\tiny, black] at ( 0.8, 0.8) {$i+1,j+1$};
  \end{tikzpicture}
  \caption{Exemplo de ordenação para um problema bidimensional.}
  \label{fig:ord_comp}
  \end{figure}
\end{frame}

%----------------------------------------------------------------------------------------

\begin{frame}{Caso bidimensional: Implementação}
  Dessa forma, com essa ordenação, a matriz terá uma estrutura pentadiagonal, três diagonais sucessivas e duas a uma distância $M$ da diagonal principal:
  \[
  A_{9\times 9} = 
  \begin{bmatrix}
    \mathsf{x} & \mathsf{x} &            & \mathsf{x} &            &            &            &            &            \\
    \mathsf{x} & \mathsf{x} & \mathsf{x} &            & \mathsf{x} &            &            &            &            \\
               & \mathsf{x} & \mathsf{x} &            &            & \mathsf{x} &            &            &            \\
    \mathsf{x} &            &            & \mathsf{x} & \mathsf{x} &            & \mathsf{x} &            &            \\
               & \mathsf{x} &            & \mathsf{x} & \mathsf{x} & \mathsf{x} &            & \mathsf{x} &            \\
               &            & \mathsf{x} &            & \mathsf{x} & \mathsf{x} &            &            & \mathsf{x} \\
               &            &            & \mathsf{x} &            &            & \mathsf{x} & \mathsf{x} &            \\
               &            &            &            & \mathsf{x} &            & \mathsf{x} & \mathsf{x} & \mathsf{x} \\
               &            &            &            &            & \mathsf{x} &            & \mathsf{x} & \mathsf{x}   
  \end{bmatrix}.
  \]
  Essa ordenação pode ser calculada por uma relação algébrica dada por
  \[
    k = i + (j - 1)M,
  \]
  onde $k$ é o número da incógnita e correspondente linha da matriz, $M$ o número de células em cada linha da malha, $i = 1,2,..., M$ e $j = 1,2,..., M$.
\end{frame}

%----------------------------------------------------------------------------------------

\begin{frame}{Caso bidimensional: Implementação}
  Nota-se que a matriz em questão é quadrada e pode ser descrita por blocos, haja vista
  \[
    \mathbf{A}_{9\times 9} = 
    \begin{bmatrix}
      A & C &         \\
      B & A & C \\
              & B & A
    \end{bmatrix},
  \]
  e portanto, dadas certas propriedades, é possível fatorá-la em duas matrizes (diagonais inferior $L$ e superior $U$) da forma
  \[
    \mathbf{A}_{9\times 9} = LU =
    \begin{bmatrix}
      \bar{A}_1 &     &     \\
      \bar{B}_2 & \bar{A}_2 &     \\
          & \bar{B}_3 & \bar{A}_3
    \end{bmatrix}
    \begin{bmatrix}
      I_1 & \Gamma_1 & \\ 
          & I_2      & \Gamma_2 \\
          &          & I_3
    \end{bmatrix}.
  \]
  Pensando em matrizes mais gerais, com $Q = M \times N$, o sistema $Aw = d$, então, teria também $w$ e $d$ em blocos
  \begin{equation}
    w = (w^{(1)}, \cdots, w^{(Q)})^t \text{ e } d = (d^{(1)}, \cdots, d^{(Q)})^t.
  \end{equation}
\end{frame}

%----------------------------------------------------------------------------------------

% \begin{frame}{Caso bidimensional: Implementação (Algoritmo)}
%   \textbf{Passo 1)} Colocar $\bar{A}_1 = A_1$ e resolver (para $\Gamma_1$):
%   \[
%     \bar{A}_1\Gamma_1 = C_1;
%   \]
%   \textbf{Passo 2)} Para $i = 2, 3, ..., Q-1$, calcular
%   \[
%     \bar{A}_i = A_i - B_i\Gamma_{i-1}
%   \]
%   e resolver (para $\Gamma_i$)
%   \[
%     \bar{A}_i\Gamma_i = C_i;
%   \]
%   \textbf{Passo 3)} Calcular $\bar{A}_N = A_N - B_N\Gamma_{Q-1}$;
  
%   \noindent\textbf{Passo 4)} Pensando em $Lz = d$, resolver $\bar{A}_1z^{(1)} = d^{(1)}$ e, para $i = 2, 3, ..., Q$,
%   \[
%     \bar{A}_iz^{(i)} = d^{(i)} - B_iw^{(i-1)};
%   \]
%   \textbf{Passo 5)} Calcular, com $w^{(Q)} = z^{(Q)}$, para $i = N-1, N-2, ..., 1$,
%   \[
%     w^{(i)} = z^{(i)} - \Gamma_iw^{(i+1)};
%   \]
%   \textbf{Passo 6)} \textbf{Saída:} As aproximações $w_i$ para $i = 1,\ldots,Q$.
% \end{frame}

%----------------------------------------------------------------------------------------

\begin{frame}{Caso bidimensional: Exemplo}
  \begin{definicao}\label{qot5}
    \justifying
    Em um reservatório, quando se secciona uma porção deste lugar em um domínio $\Omega$ com exatamente poço de injeção e outro de produção (onde se é extraído um fluido), esta porção é denominada como \textbf{a quarter of the five spot}. A injeção e produção são induzidas por um termo fonte $q$
    \[
      q = 
      \left\{
        \begin{aligned}
        \tilde{q} && &\text{no poço de injeção} \\
        -\tilde{q} && &\text{no poço de produção} \\
        0 && &\text{no restante do domínio}
        \end{aligned},
      \right.
    \]
    aplicado em células localizadas em extremos de $\Omega$, geralmente a injeção no extremo inferior esquerdo e a produção no superior direito.
  \end{definicao}
\end{frame}

%----------------------------------------------------------------------------------------

\begin{frame}{Caso bidimensional: Exemplo}
  \begin{exemplo}
    \justifying
    Dado um problema elíptico bidimensional simplificado como \eqref{escoamento_base}, uma configuração \textbf{a quarter of the five spot} como na definição \ref{qot5} e condições de contorno homogêneas de Neumann:
    \[
      \left\{
        \begin{aligned}
        \nabla \cdot u &= q && \text{em $\Omega = [0,1]\times [0,1]$} \\
        u\cdot n &= 0 && \text{sobre $\partial\Omega$} \\
        -K \nabla p &= u && \text{(Velocidade de Darcy)}
        \end{aligned}
      \right..
    \] 
    Com uma permeabilidade absoluta $K(x) = 1$ constante, termo fonte $\tilde{q} = 1$, poço de injeção na célula $(1,1)$ e poço de produção em $(N,M)$.

    Para resolver com exatidão o problema, foi usado um recurso de \textbf{contorno Dirichlet local}, onde se impõe uma pressão em um ponto específico da matriz, nesse caso, o ponto $(1,1)$ por simplicidade.
  \end{exemplo}
\end{frame}

%----------------------------------------------------------------------------------------

\begin{frame}{Caso bidimensional: Implementação}
  \begin{exemplo}    
    \begin{figure}[H]
      \centering
      \includegraphics[width=.55\textwidth]{img/desenvolvimento_2Deliptica_ex2.png}
      \caption{Campo de pressões e vetores do exemplo.}
    \end{figure}
  \end{exemplo}
\end{frame}

%----------------------------------------------------------------------------------------
\section{Introdução a problemas de transporte passivo em meios porosos} % Seções são adicionadas para organizar sua apresentação em blocos discretos, todas as seções e subseções são automaticamente exibidas no índice como uma visão geral da apresentação, mas NÃO são exibidas como slides separados.

%----------------------------------------------------------------------------------------

\begin{frame}{Problemas de transporte passivo em meios porosos}
	O movimento de um fluido em escoamento monofásico em meios porosos é descrito por um \textit{problema de transporte passivo}, onde o \textbf{fluido marcado} segue o escoamento \textit{sem alterar suas propriedades}.

	Com a velocidade do fluido $u$ obtida através da solução do problema elíptico dado por \eqref{escoamento_base}\ e \eqref{leiDeDarcy}, esse deslocamento pode ser estimado pela seguinte equação hiperbólica:
	\begin{equation}\label{transpPassivCompleto}
	\left\{
	  \begin{aligned}
	    \frac{\partial}{\partial t}(\phi c) + \nabla \cdot (uc) &= q && \text{em $\Omega$} \\
	    c(x,t=0) &= c_0 && \text{em $\Omega$} \\
	    c(x,t) &= c_b(x,t) && \text{em $\partial\Omega^-$}
	  \end{aligned},
	\right.
	\end{equation}
	onde $\phi$ é a porosidade, $q$ é o termo fonte, $c(x,t)$ é a concentração do contaminante, $c_0$ é a condição inicial e $c_b$ é a concentração nas bordas de entrada
	\[
		\partial\Omega^- = \{x \in \partial\Omega; u \cdot n < 0\},	
	\]
	onde $n$ é a normal exterior à fronteira $\partial\Omega$.
\end{frame}

%----------------------------------------------------------------------------------------

\begin{frame}{Problemas de transporte passivo em meios porosos}
	Em simulações numéricas, geralmente o termo fonte da equação \eqref{transpPassivCompleto}\ leva em conta os poços de injeção e produção, os quais podem ser convertidos em condições de contorno adequadas, gerando a equação:

	\begin{equation}
	\left\{
	  \begin{aligned}
	   \phi \frac{\partial c}{\partial t} + \nabla \cdot (uc) &= 0 && \text{em $\Omega$} \\
	    c(x,t=0) &= c_0 && \text{em $\Omega$} \\
	    c(x,t) &= c_b(x,t) && \text{em $\partial\Omega^-$}
	  \end{aligned},
	\right.
	\end{equation}
	onde a porosidade $\phi = \phi(x)$ é constante no tempo. Caso a porosidade seja uniforme e constante no tempo, é possível escalonar a primeira equação de (\theequation) para
	\begin{equation}\label{eq53}
	  \frac{\partial c}{\partial \tau} + \nabla \cdot (uc) = 0,
	\end{equation}
	onde $\tau = t/\phi$. Admitindo que a velocidade $u$ é conhecida e não depende da concentração (considerando que $K$ é constante), têm-se uma \textbf{lei de conservação hiperbólica linear}.
\end{frame}

%----------------------------------------------------------------------------------------

\begin{frame}{Derivação de leis de conservação hiperbólicas}
	Aqui, deriva-se uma \textbf{lei de balanço} para determinar a conservação de \textit{concentração} $c(x,t)$ em um domínio $\Omega \subset \mathbb{R}^n$. Essa lei de balanço estabelece que a variação temporal da quantidade $c$ em um domínio $\Omega$ é igual a taxa de fluxo de $c$ por $\partial \Omega$ mais o total de $c$ injetado ou retirado de $\Omega$:
	\begin{equation}
	  \frac{\partial}{\partial t}\int_\Omega\phi c(x,t)\ dx = - \int_{\partial\Omega}f(c(x,t))\cdot n\ ds + \int_\Omega q\ dx,
	\end{equation}
	onde $n$ é o vetor normal à $\partial\Omega$, $f(c)$ é a função de fluxo dependendo de $c$ (não necessariamente linear) e $q$ o termo fonte. Com o teorema da divergência:
	\begin{equation}
	  \int_\Omega\left(\frac{\partial}{\partial t}(\phi c(x,t)) + \nabla \cdot f(c(x,t))- q\right)\ dx = 0.
	\end{equation}
	Como a equação (\theequation) vale para qualquer domínio arbitrário $\Omega$ e a porosidade $\phi = \phi(x)$ é constante no tempo, então é possível obter a forma diferencial:
	\begin{equation}\label{EDPH}
	  \phi \frac{\partial c}{\partial t} + \nabla \cdot f(c) = q
	\end{equation}
	chamada, então, de \textbf{equação diferencial parcial hiperbólica}.
\end{frame}
\section{Método de volumes finitos para equações hiperbólicas} % Seções são adicionadas para organizar sua apresentação em blocos discretos, todas as seções e subseções são automaticamente exibidas no índice como uma visão geral da apresentação, mas NÃO são exibidas como slides separados.

%----------------------------------------------------------------------------------------

\begin{frame}{Método de volumes finitos para equações hiperbólicas}
	Considerando uma lei de conservação hiperbólica da forma \eqref{EDPH}\ com $\phi = 1$ e $q = 0$, têm-se a forma unidimensional
	\begin{equation}\label{LeideConsevacaoMVF}
	  \frac{\partial c}{\partial t} + \frac{\partial}{\partial x}f(c) = 0.
	\end{equation}
	% Por simplicidade, a discretização será uniforme e centrada nos pontos: $x_i$, $i = 1,...,N$, de forma que as interfaces entre dois volumes de controle $V_{i-1}$ e $V_i$ são dadas por
	% \[
	%   x_{i\pm \frac{1}{2}} = x_i \pm \frac{\Delta x}{2}.
	% \]
	% Com isso, cada volume de controle é dado por
	% \[
	%   V_i = [x_{i-\frac{1}{2}},x_{i+\frac{1}{2}}).
	% \]
	% A discretização temporal também será considerada uniforme, com cada passo de tempo de tamanho $\Delta t$, sendo cada nível denotado por $t^n = n\Delta t$.
	Em cada $t^n$, definine-se:
	\[
	  C_i^n = \frac{1}{\Delta x}\int_{x_{i-\frac{1}{2}}}^{x_{i+\frac{1}{2}}}c(x,t^n)\ dx \quad \text{ e } \quad
		\bar{F}_{i\pm\frac{1}{2}}^n = \frac{1}{\Delta t}\int_{t^n}^{t^{n+1}}f(c(x_{i\pm\frac{1}{2}},t))\ dt.
	\]
	Integrando a lei de conservação \eqref{LeideConsevacaoMVF} em $[x_{i-\frac{1}{2}},x_{i+\frac{1}{2}}) \times [t^n,t^{n+1})$, separando as integrais e utilizando o teorema fundamental do cálculo, têm-se:
	\begin{equation}\label{eqDaConcentracao}
	  C_i^{n+1} = C_i^n - \frac{\Delta t}{\Delta x}\left(\bar{F}_{i+\frac{1}{2}}^n - \bar{F}_{i-\frac{1}{2}}^n\right).
	\end{equation}
	Ou seja, a equação acima estabelece um princípio de conservação: \textit{a variação média da concentração na célula é dada pela diferença dos fluxos nas fronteiras da mesma}; ainda sem quaisquer tipos de aproximações.
\end{frame}

%----------------------------------------------------------------------------------------

% \begin{frame}{Método de volumes finitos para equações hiperbólicas}
% 	A representação pode ser dada por uma malha tal como:
% 	\begin{figure}[H]
% 	\centering
% 	\begin{tikzpicture}[scale=1]
% 	  \draw[black] (-0.3,-1.0) -- (3.3,-1.0);
% 	  \draw[black] (-0.3, 0.0) -- (3.3, 0.0);
% 	  \draw[black] (-0.3, 1.0) -- (3.3, 1.0);

% 	  \draw[black] (0.0,-1.3) -- (0.0,1.3);
% 	  \draw[black] (1.0,-1.3) -- (1.0,1.3);
% 	  \draw[black] (2.0,-1.3) -- (2.0,1.3);
% 	  \draw[black] (3.0,-1.3) -- (3.0,1.3);

% 	  \draw[black] (1.0,-1.3) node[below] {${x_{i-1/2}}$};
% 	  \draw[black] (2.0,-1.3) node[below] {${x_{i+1/2}}$};
% 	  \draw[black] (3.3,-1.0) node[right] {${t^{n}}$};
% 	  \draw[black] (3.3, 0.0) node[right] {${t^{n+1}}$};

% 	  \draw[black] (0.5,-0.5) node {$\mathbf{C_{i-1}^{n}}$};
% 	  \draw[black] (1.5,-0.5) node {$\mathbf{C_{i}^{n}}$};
% 	  \draw[black] (2.5,-0.5) node {$\mathbf{C_{i+1}^{n}}$};
% 	  \draw[black] (1.5, 0.5) node {$\mathbf{C_{i}^{n+1}}$};
% 	\end{tikzpicture}
% 	%\caption{Representação da malha.}
% 	\label{fig:discrLCH}
% 	\end{figure}
% 	Em cada $t^n$, a aproximação da solução no volume de controle é dada pelo valor médio de concentração nessa célula:
% 	\[
% 	  C_i^n = \frac{1}{\Delta x}\int_{x_{i-\frac{1}{2}}}^{x_{i+\frac{1}{2}}}c(x,t^n)\ dx,
% 	\]
% 	e também, define-se uma média temporal da função fluxo
% 	\[
% 	  \bar{F}_{i\pm\frac{1}{2}}^n = \frac{1}{\Delta t}\int_{t^n}^{t^{n+1}}f(c(x_{i\pm\frac{1}{2}},t))\ dt.
% 	\]
% \end{frame}

%----------------------------------------------------------------------------------------

% \begin{frame}{Método de volumes finitos para equações hiperbólicas}
% 	Integrando a lei de conservação \eqref{LeideConsevacaoMVF} em $[x_{i-\frac{1}{2}},x_{i+\frac{1}{2}}) \times [t^n,t^{n+1})$:
% 	\[
% 	  \int_{t^n}^{t^{n+1}}\int_{x_{i-\frac{1}{2}}}^{x_{i+\frac{1}{2}}}\left(\frac{\partial c}{\partial t} + \frac{\partial}{\partial x}f(c)\right)\ dx\ dt = 0,
% 	\]
% 	separando as integrais e utilizando o teorema fundamental do cálculo, têm-se:
% 	\begin{equation}\label{eqDaConcentracao}
% 	  C_i^{n+1} = C_i^n - \frac{\Delta t}{\Delta x}\left(\bar{F}_{i+\frac{1}{2}}^n - \bar{F}_{i-\frac{1}{2}}^n\right).
% 	\end{equation}
% 	Ou seja, a equação acima estabelece um princípio de conservação: \textit{a variação média da concentração na célula é dada pela diferença dos fluxos nas fronteiras da mesma}; ainda sem quaisquer tipos de aproximações.
% \end{frame}

%----------------------------------------------------------------------------------------

\begin{frame}{Método upwind para aproximação de fluxos discretos}
	Para aproximar os fluxos discretos $F^n_{i\pm 1/2}$, pode-se usar diversos métodos, como:
	\begin{itemize}
		\item \textit{Esquema central} (diferenças finitas);
		\item Método de \textit{Lax-Friedrichs}; ou
		\item Método de \textit{Lax-Wendroff}.
	\end{itemize}

	O método usado neste trabalho será o \textbf{método upwind}, que leva em conta a \textit{estrutura da solução}, de modo que a informação em cada ponto é obtida olhando a direção na qual a mesma se propaga.
		\begin{figure}[H]
	\centering
	\begin{tikzpicture}[scale=1]
	  \draw[black] (-0.3,-1.0) -- (3.3,-1.0);
	  \draw[black] (-0.3, 0.0) -- (3.3, 0.0);

	  \draw[black] (0.0,-1) -- (0.0,1);
	  \draw[black] (1.0,-1) -- (1.0,1);
	  \draw[black] (2.0,-1) -- (2.0,1);
	  \draw[black] (3.0,-1) -- (3.0,1);

	  \draw[black] (3.3,-1.0) node[right] {${t^{n}}$};
	  \draw[black] (3.3, 0.0) node[right] {${t^{n+1}}$};

	  \draw[black, font=\small] (0.5,-1.3) node {$\mathbf{C_{i-1}^{n}}$};
	  \draw[black, font=\small] (1.5,-1.3) node {$\mathbf{C_{i}^{n}}$};
	  \draw[black, font=\small] (2.5,-1.3) node {$\mathbf{C_{i+1}^{n}}$};
		
		\foreach \x in {1.0,1.2,1.4,1.6,1.8,2.0} {
	    \draw[->, red] (\x+.4,-1.0) -- (\x+.8,0.0);
	  }
	\end{tikzpicture}
	\caption{Propagação da informação em uma célula.}
	\label{fig:discrLCH}
	\end{figure}
\end{frame}

%----------------------------------------------------------------------------------------

\begin{frame}{Método upwind: Caso unidimensional}
	Considerando $u^+ = \text{max}\{u,0\}$ e $u^- = \text{min}\{u,0\}$, é possível discretizar \eqref{eqDaConcentracao} no caso de \textbf{velocidades constantes}:
	\begin{equation}\label{eqUWVelConstante}
	  \begin{aligned}
	    C_i^{n+1} &= C_i^n - \frac{\Delta t}{\Delta x}\left(u^+C^n_{i} + u^-C^n_{i+1} - u^+C^n_{i-1} - u^-C^n_{i}\right).
	  \end{aligned}
	\end{equation}
	Agora, no caso de \textbf{velocidades variáveis}:
	\begin{equation}\label{eqUWVelVariadas}
		\begin{aligned}
		C_i^{n+1} &= C_i^n - \frac{\Delta t}{\Delta x} \left[ \left( u^+(x_{i+\frac{1}{2}}) - u^+(x_{i-\frac{1}{2}}) + u^-(x_{i+\frac{1}{2}}) - u^-(x_{i-\frac{1}{2}}) \right) C_i^n \right] \\
		&- \frac{\Delta t}{\Delta x} \left[ u^-(x_{i-\frac{1}{2}}) W_{i-\frac{1}{2}} + u^+(x_{i+\frac{1}{2}}) W_{i+\frac{1}{2}} \right].
		\end{aligned}
	\end{equation}
	Também, é preciso satisfazer a \textbf{condição CFL}, uma condição necessária, mas não suficiente, para garantir convergência do método de volumes finitos para a equação diferencial. A convergência ocorre caso:
	\[
	  u\frac{\Delta t}{\Delta x} \leq 1,
	\]
	com $u$ constante.
\end{frame}

%----------------------------------------------------------------------------------------

\begin{frame}{Método upwind: Exemplo do caso unidimensional}
	\begin{exemplo}
		\justifying
		Dado o problema de contorno unidimensional \eqref{escoamento_base}, que modela o escoamente monofásico de um reservatório saturado, com sistema de equações dado por:
	  \begin{equation}\label{Elip282}
	    \left\{
	      \begin{aligned}
	      -\frac{d}{dx}\left(K\frac{dp}{dx}\right) &= -25\cos(25x) && \text{em $\Omega_e = [0,1]$} \\
	      p &= x && \text{sobre $\partial\Omega_e$}
	      \end{aligned}
	    \right.
	  \end{equation}
	  onde $K(x) = 2 + \sin(25x)$. Será simulado o fluxo de um contaminante com concentrações $c(x,t)$, governado pelo sistema:
	  \[
	  \left\{
	    \begin{aligned}
	      \partial_t c + u\partial_xc &= 0 && \text{em $\Omega_h$} \\
	      c_0(x) &= e^{-20(3x-.5)^2} + e^{-(3x-3.5)^2}&& \text{em $\Omega_h$}
	    \end{aligned}
	  \right.,
	  \]
	  Utilizando a equação \eqref{eqUWVelVariadas}, é possível chegar numa solução visualizada pelos gráficos a seguir:
	\end{exemplo}
\end{frame}

%----------------------------------------------------------------------------------------

\begin{frame}{Método upwind: Exemplo do caso unidimensional}
	\begin{exemplo}
		\centering
	  \animategraphics[autoplay,loop,width=0.8\linewidth]{30}{1dVFUW_frames_temp/frame_}{000}{100}
	\end{exemplo}
\end{frame}

%----------------------------------------------------------------------------------------

\begin{frame}{Método upwind: Caso bidimensional}
	Para problemas em duas dimensões, a lei de conservação \eqref{LeideConsevacaoMVF} assume a forma
	\begin{equation}\label{LC2D}
	  c_t + f_x(c) + g_y(c) = 0,
	\end{equation}
	onde a concentração do fluido depende de $x$, $y$ e $t$. Definindo
	\begin{equation}
	  \bar{F}_{i+\frac{1}{2},j}^{n} = \frac{1}{\Delta t \Delta y} \int_{t^{n}}^{t^{n+1}} \int_{y_{j-\frac{1}{2}}}^{y_{j+\frac{1}{2}}} f(c(x_{i+\frac{1}{2}}, y, t)) \, dy \, dt
	\end{equation}
	e
	\begin{equation}
	  \bar{G}_{i,j+\frac{1}{2}}^{n} = \frac{1}{\Delta t \Delta x} \int_{t^{n}}^{t^{n+1}} \int_{x_{i-\frac{1}{2}}}^{x_{i+\frac{1}{2}}} g(c(x, y_{j+\frac{1}{2}}, t)) \, dx \, dt,
	\end{equation}
	têm-se
	\begin{equation}
	  C_{i,j}^{n+1} = C_{i,j}^{n} - \frac{\Delta t}{\Delta x} \left( \bar{F}_{i+\frac{1}{2},j}^{n} - \bar{F}_{i-\frac{1}{2},j}^{n} \right) - \frac{\Delta t}{\Delta y} \left( \bar{G}_{i,j+\frac{1}{2}}^{n} - \bar{G}_{i,j-\frac{1}{2}}^{n} \right)
	\end{equation}
	onde os fluxos $\bar{F}$ e $\bar{G}$ podem ser aproximados por fluxos discretos em cada direção, assim como em \eqref{eqUWVelConstante}\ ou \eqref{eqUWVelVariadas}.
\end{frame}

%----------------------------------------------------------------------------------------

\begin{frame}{Método upwind: Exemplo do caso bidimensional}
	\begin{exemplo}\label{ex2CB}
		O primeiro exemplo é o mais simples: um problema elíptico bidimensional simplificado como \eqref{escoamento_base}, uma configuração \textit{a quarter of the five spot} como na definição \ref{qot5} e condições de contorno homogêneas de Neumann:
	  \[
	    \left\{
	      \begin{aligned}
	      \nabla \cdot u &= q && \text{em $\Omega = [0,1]\times [0,1]$} \\
	      u\cdot n &= 0 && \text{sobre $\partial\Omega$} \\
	      -K \nabla p &= u && \text{(Velocidade de Darcy)}
	      \end{aligned}
	    \right..
	  \] 
	  Com uma permeabilidade absoluta $K(x,y) = 1$ constante, termo fonte $\tilde{q} = 1$, poço de injeção na célula $(1,1)$ e poço de produção em $(N,M)$.
	\end{exemplo}
\end{frame}

%----------------------------------------------------------------------------------------

\begin{frame}
	\begin{exemplo}
		\centering
	  \animategraphics[autoplay,loop,width=0.75\linewidth]{30}{1ex2dVFUW_frames_temp/frame_}{0000}{0100}
	\end{exemplo}
\end{frame}

%----------------------------------------------------------------------------------------
\begin{frame}{Método upwind: Exemplo do caso bidimensional}
	\begin{exemplo}\label{ex2CB}
		\justifying
	  Neste segundo exemplo, dado um problema elíptico bidimensional simplificado como \eqref{escoamento_base}, uma configuração \textit{a quarter of the five spot}, campo de permeabilidade e de pressões e velocidades visualizados abaixo:
	  \begin{figure}[H]
	    \centering
	    \begin{subfigure}[c]{0.4\textwidth}
	      \centering
	      \includegraphics[width=\textwidth]{img/ex2_perm_hetero.png}
	    \end{subfigure}
	    \begin{subfigure}[c]{0.4\textwidth}
	      \centering
	      \includegraphics[width=\textwidth]{img/ex2_pressao_velocidade.png}
	    \end{subfigure}
	    \caption{Gráficos de permeabilidade e de pressão com velocidades não normalizadas.}
	  \end{figure}
	\end{exemplo}
\end{frame}

%----------------------------------------------------------------------------------------

\begin{frame}
	\begin{exemplo}
		\centering
	  \animategraphics[autoplay,loop,width=0.75\linewidth]{30}{2ex2dVFUW_frames_temp/frame_}{0000}{0100}
	\end{exemplo}
\end{frame}

%----------------------------------------------------------------------------------------
\section{Referências} % Seções são adicionadas para organizar sua apresentação em blocos discretos, todas as seções e subseções são automaticamente exibidas no índice como uma visão geral da apresentação, mas NÃO são exibidas como slides separados.

\begin{frame}[allowframebreaks]{Referências}
  \nocite{*}
  \justifying
  \printbibliography[heading=none]
\end{frame}

% Slide final
\begin{frame}
    \begin{center}
        {\Huge Obrigado!}
    \end{center}
\end{frame}

\end{document}